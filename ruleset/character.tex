\chapter{Character}

This chapter describes \emph{Characters}. Currently this chapter describes only
meta-human \emph{Characters} with a physical body to be played by a \emph{Player}.
In principle, certain types of \emph{Agents} and \emph{Spirits} could also be played,
but are currently not in scope of this rule-set.
body in particular.

\section{Attributes}
\index{Atribute}

\emph{Attributes} are very central values in defining a \emph{Character's}
abilities. They give a broad description of a \emph{Character's} strengths and
weaknesses and are influencing both final \emph{Skill} values as well as derived
\emph{Characteristics}.

The base value for most \emph{Attributes} of an average human is 8.

\begin{table}[htb]
    \rowcolors{1}{}{lightgray}
    \caption[Attribute Values]{Attribute Values}
    \label{tab:attribute values}
    \centering
    \begin{tabular}{cl}
        \toprule
        \textbf{Value} & \textbf{Description} \\
        \midrule
        \( \le 3 \)    & Disabled             \\
        4--5           & Challenged           \\
        6--7           & Underdeveloped       \\
        8              & Average              \\
        9--10          & Improved             \\
        11--12         & Superior             \\
        13--14         & Exemplar             \\
        \( \ge 15 \)   & Superhuman           \\
        \bottomrule
    \end{tabular}
\end{table}

\emph{Attribute} values in Pink Trenchcoat are logarithmic with a base of 3.
This means that a \emph{Character} with \emph{Strength} 11 is twice as strong as a
\emph{Character} with \emph{Strength} 8 which in turn is twice as strong as a
\emph{Character} with \emph{Strength} 5.
This fact is only influencing certain \emph{Characteristics} like
\emph{Carrying Capacity} and does not need to be kept in mind in most situations.


\subsection{Mental Attributes}
Pink Trenchcoat uses four \emph{Mental Attributes}.
\paragraph{Charisma}
\index{Charisma}
\emph{Charisma} describes a \emph{Character's} ability to positively affect other
people in interactions. Highly charismatic people instantly get the attention of
others, are often favored, and respected. A person with a low \emph{Charisma} value is
often ignored and sometimes not taken seriously.
\emph{Charisma} is also required to connect with people emotionally and understand
emotional context of a conversation.

\emph{Ware} is negatively affecting \emph{Charisma} as it detaches the \emph{Character}
from itself.

\paragraph{Inutition}
\index{Intuition}
\emph{Intuition} describes the \emph{Character's} ability to intuitively ans
subconsciously process information. It describes not how fast or how much
the \emph{Character} can process, but how well. Furthermore a high \emph{Intuition}
value helps the \emph{Character} to grasp a situation faster and perceive better.

\paragraph{Logic}
\index{Logic}
\emph{Logic} describes the raw processing power and storage capacity of a
\emph{Characters} brain. Combined with \emph{Intuition}, both attributes form the
\emph{Characters} IQ. A high \emph{Logic} value helps with most \emph{Craftsmanship}
and all \emph{Knowledge Skills}.

\paragraph{Willpower}
\index{Willpower}
\emph{Willpower} represents the amount of control the \emph{Character} has about
their mind and body. How far they can force their body to go, and how well to
withstand temptations of any kind. It is also a measure for courage.

\subsection{Physical Attributes}

Pink Trenchcoat uses four \emph{Physical Attributes}.

\paragraph{Agility}
\index{Agility}

\emph{Agility} represents a \emph{Character's} nimbleness and dexterity. The motions
of a \emph{Character} with high \emph{Agility} look fluid and smooth, while low
\emph{Agility} motions look stocky. \emph{Agility} is important for all
\emph{Close Combat} and most \emph{Physical Skills}. Larger \emph{Characters}
normally have lower \emph{Agility}.

\paragraph{Body}
\index{Body}
Body describes a \emph{Character's} ability to endure physical strain, and keep
going, even when exhausted. It also influences how much \emph{Damage} the body can
take before collapsing. \emph{Body} is independent of \emph{Size}, meaning that a
large \emph{Character} does have the same average \emph{Body} as a smaller one.

\paragraph{Coordination}
\index{Coordination}

\emph{Coordination} is the ability to control your body the way you want, especially
hand-eye coordination. Although a \emph{Character's} body can be very agile, as long
as the \emph{character} can not control it in the right way, it may not help much.
\emph{Coordination} is important where the \emph{Character} works with his hands,
like in \emph{Ranged Combat} or most \emph{Craftsmanship Skills}.

\paragraph{Strength}
\index{Strength}

\emph{Strength} measures the raw power of a \emph{Character's} body, the pure muscle
volume. Most \emph{Physical Skills} benefit from a high \emph{Strength} value.
\emph{Strength} generally increases with \emph{Size}.

\subsection{Other Attributes}
\paragraph{Fate}
\index{Fate}

\emph{Fate} is a measure of a \emph{Character's} luck, the favour of the gods or their
balance score with the universe itself. Or it is just a gamistic resource that can
affect \emph{Tests}.

\emph{Fate} refreshes every game session and can be used to modify \emph{Test} either
before or after the roll. It can only be taken in high stake moments, that are critical
for the story or the \emph{Character}. The \emph{Game Master} decides if this is the case.

\subparagraph{Optional: Anomaly Ownership}
\index{Anomaly Ownership}
A \emph{Player} can also, by spending \emph{Fate}, take ownership of an \emph{Anomaly}
they have rolled. This means that now the \emph{Player} instead of the
\emph{Game Master} decides and describes what the special effects of the \emph{Anomaly} are.
The \emph{Game Master} however needs to accept the effect and decides how much
\emph{Fate} it costs.
\begin{table}[htb]
    \rowcolors{1}{}{lightgray}
    \caption[Fate Costs]{Fate Costs}
    \label{tab:fate costs}
    \centering
    \begin{tabular}{cl}
        \toprule
        \textbf{Value} & \textbf{Description} \\
        \midrule
        1              & +1 before            \\
        3              & +2 before            \\
        6              & +3 before            \\
        4              & +1 after             \\
        9              & +2 after             \\
        1+             & own \emph{Anomaly}   \\
        \bottomrule
    \end{tabular}
\end{table}

\paragraph{Magic}
\index{Magic (Attribute)}

\paragraph{Size}
\index{Size (Attribute)}

\section{Characteristics}

\subsection{Health}

\paragraph{Life}
\index{Life}

\paragraph{Wound Limit}
\index{Wound Limit}

\paragraph{Damage Pip}
\index{Damage Pip}

\paragraph{Wound Heal Time}
\index{Wound Heal Time}

\subsection{Athletics}
\paragraph{Carrying Capacity}
\index{Carrying Capacity}

\paragraph{Combat Speed}
\index{Combat Speed}

\paragraph{Action Costs}
\index{Action Costs}

Action Costs describe how many \emph{Ticks} a \emph{Character} needs to spend
for any kind of \emph{Action}.

\paragraph{Reaction}
\index{Reaction}

\emph{Reaction} determines the speed a \emph{Character} can react to external
events like suddenly appearing dangers or actions in combat.


\section{Skills}
\index{Skills}

\begin{table}[htb]
    \rowcolors{1}{}{lightgray}
    \caption[Skill Values]{Skill Values}
    \label{tab:skill values}
    \centering
    \begin{tabular}{cl}
        \toprule
        \textbf{Value} & \textbf{Description} \\
        \midrule
        0              & Untrained            \\
        1--3           & Amateur              \\
        4--6           & Journeyman           \\
        7--9           & Senior Journeyman    \\
        10--12         & Master               \\
        13--15         & Elite                \\
        16--18         & Legendary            \\
        \( \ge 19 \)   & Godlike              \\
        \bottomrule
    \end{tabular}
\end{table}

\subsection{Combat}
\index{Combat (Skill)}

\subsubsection{Melee Combat}
\index{Melee Combat (Skill)}

\subsubsection{Direct Weapons}
\index{Direct Weapons}

\subsubsection{Ballistic Weapons}
\index{Ballistic Weapons}

\subsection{Craftsmanship}
\index{Craftsmanship Skills}

\subsubsection{Chemistry}
\index{Chemistry}

\subsubsection{Mechanics}
\index{Mechanics}

\subsubsection{Mechatronics}
\index{Mechatronics}

\subsubsection{Medicae}
\index{Meidcae}

\subsubsection{Social Crafting}
\index{Social Crafting}


\subsection{Empathy}
\index{Empathy Skills}

\subsubsection{Discourse}
\index{Discourse}

\subsubsection{Influence}
\index{Influence}

\subsubsection{Etiquette}
\index{Etuiquette}

\subsubsection{Scrutiny}
\index{Scrutiny}


\subsection{Magic}
\index{Magic (Skill)}

\subsubsection{Assensing}
\index{Assensing}

\subsubsection{Enchantment}
\index{Enchanment}

\subsubsection{Evocation}
\index{Evocation}

\subsubsection{Invocation}
\index{Invocation}

\subsubsection{Physical}
\index{Physical Skills}

\subsubsection{Acrobatics}
\index{Acrobatics}

\subsubsection{Athletics}
\index{Athletics}

\subsubsection{Sleight of Hand}
\index{Sleight of Hand}

\subsubsection{Stealth}
\index{Stealth}


\subsection{Piloting}
\index{Piloting Skills}

\subsubsection{Anthroforms}
\index{Anthroforms}

\subsubsection{Gunnery}
\index{Gunnery}

\subsubsection{Pilot Air}
\index{Pilot Air}

\subsubsection{Pilot Ground}
\index{Pilot Ground}

\subsection{Processing}
\index{Processing Skills}

\paragraph{Cracking}
\index{Cracking}

\subsubsection{Navigation}
\index{Navigation}

\subsubsection{Perception}
\index{Perception}

\subsubsection{Software}
\index{Software}


\subsection{Resistance}
\index{Resistance Skills}

\subsubsection{Composure}
\index{Composure}

\subsubsection{Interaction}
\index{Interaction}

\subsubsection{Memory}
\index{Memory}

\subsubsection{Survival}
\index{Survival}



