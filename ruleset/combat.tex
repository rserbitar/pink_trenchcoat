\chapter{Combat}

\section{Timing}

\subsection{Resolution Order}

In Pink Trenchcoat combat is resolved by a continuous, always increasing
time value.

\paragraph{Tick}
\index{Tick}
This time value is measured in \emph{Ticks}. A \emph{Tick} is
a time measure of about 0.3 seconds.

\paragraph{Current Tick}
\index{Current Tick}
Combat continuously advances the \emph{Current Tick} that represents the the current
point in time.

\paragraph{Initiative Score}
\index{Initiative Score}
The\ emph{Initiative Score} represents the \emph{Current Tick} in which a
\emph{Character} can \emph{declare} and take \emph{Actions}.
Ties in \emph{Initiative Score} are broken by the \emph{Characters Reflex} value.

\paragraph{Phase}
\index{Phase}
A \emph{Phase} lasts 20 \emph{Ticks}. Each Phase, this means on \emph{Current Tick }
20,40,80 and so on \emph{Continuous Effects} like fire,
toxin damage and bleeding are resolved.

\paragraph{Combat Flow}
\index{Declare Action}
\index{Resolve Action}
\index{Maximum Initiative Score}
\label{par:maximum initiative score}
Characters who's \emph{Initiative Score} matches the \emph{Current Tick}
are allowed to \emph{declare} an \emph{Action}. An \emph{Action} that would
increase a \emph{Character's Initiative Score} to greater than the

\begin{equation}
    \textit{Maximum Initiative Score} = \textit{Current Tick} + 20
\end{equation}

can not be declared.

After \emph{declaring} an \emph{Action} the \emph{Characters} \emph{Initiative Score}
is immediately increased by a value depending on the \emph{Action}. If no
\emph{Interrupts} occur, the \emph{Action} is \emph{resolved}.

When all eligible \emph{Characters} have taken their \emph{Actions}, the
\emph{Current Tick} is advanced to the next meaningful value which is normally
either the next lowest \emph{Initiative Score} of a \emph{Character}, the next
\emph{Phase} if there are any \emph{Continuous Effects} to \emph{resolve} or the
\emph{Tick} a \emph{Character} in \emph{Delay} wants to \emph{act}.


\paragraph{Interrupt}
Instead or in addition to acting on their \emph{Initiative Score}, a \emph{Character}
can also chose to \emph{interrupt} another \emph{Character} after \emph{declaring} an
\emph{Action}. After a \emph{Character} declared their \emph{Action} and
increased their \emph{Initiative Score},
but before it was \emph{resolved}, the \emph{interrupting Character} can
\emph{declare} their \emph{interrupting Action} (and also immediately increase their
\emph{Initiative Score}).
The \emph{Interrupting Action} can itself be interrupted by a
\emph{Character} that has not yet declared an \emph{Action} this \emph{Tick}.

A \emph{Reflex Test} decides the order in which \emph{Characters} taking part in
the \emph{Interrupt} are \emph{resolving} their \emph{Actions}. Each \emph{Character}
receives a \emph{Modifier} equal to:

\begin{equation}
    \textit{Interrupt Modifier} = \textit{Current Tick} - \textit{Initiative Score}
\end{equation}

In addition, the following \emph{Modifiers} apply:
\begin{table}[htb]
    \rowcolors{1}{}{lightgray}
    \caption[Interrupt Modifiers]{Interrupt Modifiers}
    \label{tab:interrupt modifiers}
    \centering
    \begin{tabular}{cl}
        \toprule
        \textbf{Modifier} & \textbf{Situation}                                \\
        \midrule
        \textbf{+9}       & \emph{Overwatch Action} triggered                 \\
        \textbf{+3}       & \emph{Aiming} or \emph{Watching} \emph{Character} \\
        \textbf{-3}       & per level of \emph{Interruption}                  \\
        \bottomrule
    \end{tabular}
\end{table}

\emph{Actions} are \emph{resolved} starting from the highest \emph{Test Quality}
to the lowest. Ties are broken first by lowest \emph{Initiative Score}, then by
\emph{Reflex} value.

\subsection{Starting Combat}

Once the Game Master decides that the Game should transition from
\emph{Narrative Time} to \emph{Combat Time}, the \emph{Characters} starting
\emph{Initiative Score} needs to be determined. To do so, each Character rolls
a \emph{Reflex Test}. The negative \emph{Test Quality} determines the initial
\emph{Initiative Score}.

Apply the following \emph{Modifiers}:

\begin{table}[htb]
    \rowcolors{1}{}{lightgray}
    \caption[Surprise Modifiers]{Surprise Modifiers}
    \label{tab:surprise modifiers}
    \centering
    \begin{tabular}{cl}
        \toprule
        \textbf{Modifier} & \textbf{Character State}        \\
        \midrule
        \textbf{+6}       & Initiated first \textit{Action} \\
        \textbf{0}        & Actively anticipating combat    \\
        \textbf{-3}       & Suspicious                      \\
        \textbf{-6}       & Not expecting combat            \\
        \textbf{-9}       & Deeply involved                 \\
        \bottomrule
    \end{tabular}
\end{table}

\section{Combat Actions}

\subsection{Action Times}
To reduce cognitive load, \emph{Action Times} are separated into four categories.
Depending
on the \emph{Character} and their \emph{Attribute}values, these \emph{Actions} take
a different amount of \emph{Ticks} to fulfil.

\paragraph{Free Action}
\index{Free Action}
\emph{Free Actions} are the shortest kind of \emph{Action}. They require no or almost no thoughts
and can be executed almost immediately.

\paragraph{Simple Action}
\index{Simple Action}
A \emph{Simple Action} can still be done quickly, or be triggered by reflex, but is
not instant.

\paragraph{Half Action}
\index{Half Action}
\emph{Half Actions }are the standard basic Action.

\paragraph{Full Action}
\index{Full Action}
\emph{Actions} taking a lot of concentration or a lot of time.

\subsection{Meta-Actions}

Meta Actions are different compared to normal Actions in a sense that they do not
follow the typical scheme of Declaration, Initiative Score increase and Resolution.

\paragraph{Delay}
\index{Delay}
A \emph{delaying} Character does not have a \emph{Initiative Score}, but can chose to
\emph{act} at any future \emph{Current Tick} and thus immediately gets the
\emph{Current Tick} assigned as \emph{Initiative Score}, before \emph{declaring }
the \emph{Action}. They have to do so before
any other \emph{Character} \emph{declared} an \emph{Action} in that \emph{Tick}.
If another \emph{Character} already \emph{declared} an \emph{Action},
the \emph{delaying} \emph{Character} can still chose to \emph{act},
but will need to perform an \emph{Interrupt}.

After performing any \emph{Action}, \emph{delay} ends.

\paragraph{Watch}
\index{Watch}
When \emph{declaring Watch} a \emph{Character} has to chose a suitable
\emph{Character} or object to \emph{watch}. The \emph{Character} immediately loses
their \emph{Initiative Score}. The watching Character is granted a bonus of

A \emph{Character} can switch from \emph{Watch} to \emph{Delay} at any time.

\paragraph{Overwatch}
\index{Overatch}
When \emph{declaring Overwatch} a \emph{Character} has to chose both a
\emph{specific Condition} and a \emph{specific Action}. The \emph{Character}
immediately loses their \emph{Initiative Score}.

Once the \emph{specific Condition} is fulfilled, even if in \emph{declaration} the
\emph{Character} immediately performs the \emph{specific Action}. The \emph{Character}
is assigned an \emph{Initiative Score} according to the \emph{Current Tick}
plus the value depending on the \emph{Action} and \emph{Overwatch} ends.
If \emph{specific Condition} was declared by another \emph{Character} an
\emph{Interrupt} is triggered. The \emph{Character} on \emph{Overwatch} receives
a +9 \emph{Modifier} for the \emph{Reflex Test} in this case.

\emph{Specific Conditions} should not be too complex and should be easily
identified as true by the \emph{Character} on \emph{Overwatch} with the current
situation.

\emph{Specific Actions} can only be \emph{Active Actions}.

A \emph{Character} can switch from \emph{Overwatch} to \emph{Delay} at any time.

\paragraph{Move}
\index{Move Action}
Normal movement, like crawling, walking or running, can be performed in addition
to \emph{Active Actions}. However, if of another \emph{Action} is performed while
moving, a \emph{Movement Modifier} is applied.

The \emph{Movement Modifier}, as well as the distance moved depend both on the
movement style as well as the duration of the movement, which is represent by the
\emph{Action Time}. The Character moves a number of meters according to their
relevant \emph{Movement Rate} and \emph{Action Time}.

A \emph{Move Action} can only performed in addition to an \emph{Active Action}
if the \emph{Action} does not include movement.

\begin{table}[htb]
    \rowcolors{1}{}{lightgray}
    \caption[Movement Modifiers]{Movement Modifiers}
    \label{tab:movement modifiers}
    \centering
    \begin{tabular}{cl}
        \toprule
        \textbf{Modifier} & \textbf{Movement Style} \\
        \midrule
        \textbf{-6}       & Crawl                   \\
        \textbf{-1}       & Walk                    \\
        \textbf{-3}       & Run                     \\
        \bottomrule
    \end{tabular}
\end{table}

\paragraph{Talk}
\index{Talk Action}

When a \emph{Character} is talking wile performing a \emph{Test}, the \emph{Test}
receives an additional -1 \emph{Modifier}.
\begin{table}[htb]
    \rowcolors{1}{}{lightgray}
    \caption[Talk]{Talk}
    \label{tab:talk}
    \centering
    \begin{tabular}{cl}
        \toprule
        \textbf{Action} & \textbf{Words}    \\
        \midrule
        \textbf{Free}   & 1 Word            \\
        \textbf{Simple} & Short sentence    \\
        \textbf{Half}   & 2 Short sentences \\
                        & 1 Long sentence   \\
        \textbf{Full}   & 4 short sentences \\
                        & 2 Long sentences  \\
        \bottomrule
    \end{tabular}
\end{table}

\paragraph{Ongoing Actions}
\index{Onging Action}

\emph{Ongoing Actions} represent \emph{Actions} that take, longer than 20
\emph{Ticks}, sometimes considerably longer. They are represented by increasing
the \emph{Initiative Score} of the \emph{Character} performing the \emph{Action}
by 20 if the \emph{Character} wants to continue doing the \emph{Ongoing Action},
till it is finished.

\subsection{Continuous Inclusive Actions}

\paragraph{Leadership}

\paragraph{Tactics}

\subsection{Active Actions}
\index{Active Action}

\emph{Active Actions} are \emph{Actions} a \emph{Character} \emph{declares} when
their \emph{Initiative Score} matches the \emph{Current Tick}. They are the
default action type.

\subsubsection{General Actions}

\paragraph{Free Actions}
\index{Free Action}
Free Actions without specific rules include:
\begin{itemize}[parsep=0em]
    \item dropping an object
    \item change facing 90°
\end{itemize}

\paragraph{Simple Actions}
\index{Simple Action}
Simple Actions without specific rules include:
\begin{itemize}[parsep=0em]
    \item \item change facing 180°
\end{itemize}

\paragraph{Half Actions}
\index{Half Action}
Half Actions without specific rules include:
\begin{itemize}[parsep=0em]
    \item change stance
\end{itemize}

\paragraph{Full Actions}
\index{Full Action}
Full Actions without specific rules include:
\begin{itemize}[parsep=0em]
    \item perform a \emph{Perception Test}
    \item perform a \emph{Jump Test}
\end{itemize}


\myparagraph{Change Posture}
\index{Change Posture}
\begin{tabular}{rl}
    \textbf{Action} & Half              \\
    \textbf{Test}   & None/Acrobatics-6 \\
\end{tabular}

\hfill

Change posture from:
\begin{itemize}[parsep=0em]
    \item standing to crouching
    \item crouching to prone
    \item standing to prone
    \item prone to crouching
    \item crouching to standing
\end{itemize}

Changing two levels at once (in this case only prone to standing)
requires an \emph{Acrobatics -6 Test}.

\myparagraph{Drop Down}
\index{Drop Down}
\begin{tabular}{rl}
    \textbf{Action} & Free \\
    \textbf{Test}   & None \\
\end{tabular}

\hfill

Dropping down from a standing posture. After doing so, the \emph{Character} is
\emph{disoriented}.

\myparagraph{Ready Weapon}
\index{Ready Weapon}
\begin{tabular}{rl}
    \textbf{Action} & var.           \\
    \textbf{Test}   & None/Quickdraw \\
\end{tabular}

\hfill

Ready a weapon given that the weapon is carried on the \emph{Character}.
A weapon needs to be ready to perform any \emph{Actions} with it.

\begin{table}[htb]
    \rowcolors{1}{}{lightgray}
    \caption[Ready Weapon]{Ready Weapon}\label{tab:ready weapon}
    \centering
    \begin{tabular}{llc}
        \toprule
        \textbf{Weapon}   & \textbf{Action} & \textbf{Test} \\
        \midrule
        \textbf{2 handed} & Full            & None          \\
        \textbf{2 handed} & Half            & Quickdraw -8  \\
        \textbf{2 handed} & Simple          & Quickdraw -16 \\
        \textbf{1 handed} & Half            & None          \\
        \textbf{1 handed} & Simple          & Quickdraw -8  \\
        \textbf{1 handed} & Free            & Quickdraw -16 \\
        \bottomrule
    \end{tabular}
\end{table}


\myparagraph{Reload Weapon}
\index{Reload Weapon}
\begin{tabular}{rl}
    \textbf{Action} & var. \\
    \textbf{Test}   & None \\
\end{tabular}

\hfill

Reloads a weapon given that ammunition is carried on the \emph{Character}.
Reload time depends on the weapon.

\myparagraph{Sprint}
\index{Sprint}
\begin{tabular}{rl}
    \textbf{Action} & Full    \\
    \textbf{Test}   & Running \\
\end{tabular}

\hfill

Move number of meters equal to the \emph{Characters} \emph{Full Running Rate}.
Add 1 meter for every 2 \emph{Test Quality } of the \emph{Running Test}.


\myparagraph{Zigzag}
\index{Zigzag}

\begin{tabular}{rl}
    \textbf{Action} & Full \\
    \textbf{Test}   & None \\
\end{tabular}
The \emph{Character} moves a number of meters equal to half their
\emph{Running Rate}. All attackers get the “Running 90° to Attacker”
\emph{Modifier}. Using the \emph{Zigzag Action} is not possible when
carrying an \emph{unwieldy} weapon and not exceeding its
\emph{Required Strength} by 3.

\subsubsection{Ranged Actions}
\index{Ranged Action}

\emph{Ranged Actions} are specific to ranged combat.

\myparagraph{Aim}
\begin{tabular}{rl}
    \textbf{Action} & Half/Full \\
    \textbf{Test}   & None      \\
    \textbf{Damage} & No        \\
\end{tabular}

\hfill

The \emph{Character} aims at a target in their \emph{Front} or \emph{Side Arc}.
The gain a +1 \emph{Modifier} for all \emph{Ranged Actions} that do
\emph{Damage} on the aimed target. This Action can be taken multiple times till a
maximum Modifier of +4 is reached.
\emph{Aiming} with a \emph{Full Action} counts as \emph{aiming} 3 times.

Equipment modifies aiming as follows:

\begin{itemize}[parsep=0em]
    \item A Red Dot/Smartgun conveys a +2 \emph{Modifier} for the
          first \emph{Aim Action}
    \item A Rangefinder conveys a +2 \emph{Modifier} for the second
          \emph{Aim Action}
    \item A Scope allows for a maximum \emph{Modifier} of +6
    \item A Gun Light reduces emph{Visual Modifiers} for the target if the
          target is in range
\end{itemize}

The \emph{Aim Modifier} is lost when the target moves out of
sight or the \emph{Character} takes any other \emph{Action} than:

\begin{itemize}[parsep=0em]
    \item any Free Action
    \item any Ranged Action
    \item any Processing Test
    \item any Empathy Test
    \item Talk
    \item Walk
\end{itemize}

\paragraph{Brace Weapon}
\index{Brace Weapon}
\begin{tabular}{rl}
    \textbf{Action} & Full/Half \\
    \textbf{Test}   & None      \\
    \textbf{Damage} & No        \\
\end{tabular}

\hfill

A \emph{Character} braces their two handed ranged weapon against a suitable
object like a sandbag a window sill or a railing. The \emph{Weapon Range} is
doubled for all \emph{Ranged Attacks} against targets in the \emph{Characters}
\emph{Front Arc} with a height difference no greater than 1/4th of the distance.

The \emph{Required Strength} of the weapon is also reduced by 3.

If the weapon is equipped with a bipod, bracing takes only a \emph{Half Action}
and the height difference can be 1/3rd of the distance.

Weapons equipped with a tripod take a \emph{Full Action} to \emph{brace},
which means they are effectively deployed. In this case \emph{bracing} is not
lost when the \emph{Character} moves, if the \emph{Character} choses to let go of
their weapon. \emph{Changing Facing} and \emph{Posture} is also possible without
losing the \emph{braced} effect. The maximum height difference is 1/2nd.

Bracing for non tripod-weapons is lost when the \emph{Character} moves in any way
(also as a result of a \emph{Dodge}, changes facing or changes posture).

\myparagraph{Fast Ranged Attack}
\index{Fast Ranged Attack}

\begin{tabular}{rl}
    \textbf{Action} & Half              \\
    \textbf{Test}   & Ranged Weapons -3 \\
    \textbf{Damage} & Yes               \\
\end{tabular}

\hfill

The \emph{Character} performs a \emph{Ranged Attack} with a -3 \emph{Modifier}.
Weapons with the \emph{Single Shot} keyword can not perform this \emph{Action}.


\myparagraph{Burst Ranged Attack}
\index{Burst Ranged Attack}

\begin{tabular}{rl}
    \textbf{Action} & Full                 \\
    \textbf{Test}   & Ranged Weapons +2/+3 \\
    \textbf{Damage} & Yes                  \\
\end{tabular}

\hfill

The \emph{Character} performs a \emph{Ranged Attack} with a weapon that has a
\emph{Short Burst} or \emph{Long Burst} value. Short Bursts add a +2
\emph{Modifier}, Long Bursts a +3 \emph{Modifier}. For \emph{Damage Resolution}
multiply both \emph{Armor} of the target and \emph{Penetration} of the weapon
with 2 (Short Burst) or 3 (Long Burst).

Burst \emph{Ranged Attacks} can have an increased \emph{Required Strength}.

\myparagraph{Multi Ranged Attack}
\index{Multi Ranged Attack}

\begin{tabular}{rl}
    \textbf{Action} & Full                             \\
    \textbf{Test}   & Ranged Weapons -2/-1 per attack, \\
                    & -1 per additional target         \\
    \textbf{Damage} & Yes                              \\
\end{tabular}

\hfill

The \emph{Character} is performing multiple \emph{Ranged Attacks} against one or
more targets. Only one \emph{Test} is rolled by the \emph{Character},
\emph{modified} by -2 per attack and -1 per additional target.
This value is opposed by all targets individually.

This \emph{Action} is only possible with one handed ranged weapons that do
not have the \emph{Single Shot} keyword. Wielding two ranged weapons decreases
the attack \emph{Modifier} to -1 and allow for 2 attacks even with
\emph{Single Shot} weapons.

\myparagraph{Ranged Attack}
\index{Ranged Attack}
\begin{tabular}{rl}
    \textbf{Action} & Full              \\
    \textbf{Test}   & Ranged Weapons +0 \\
    \textbf{Damage} & Yes               \\
\end{tabular}

\hfill

The \emph{Character} performs a standard \emph{Ranged Attack}.


\myparagraph{Suppression Fire}
\index{Suppression Fire}
\begin{tabular}{rl}
    \textbf{Action} & Full \\
    \textbf{Test}   & No   \\
    \textbf{Damage} & No   \\
\end{tabular}

\hfill

The \emph{Characters} suppresses an area with \emph{Suppression Fire}.
The attacker can cover a length of meters equal to the number of bullets of the
weapons \emph{Suppression Fire} value. Everybody moving through the suppressed
zone or not going prone/into cover while being in it is hit with a final
\emph{Test Quality} of d10. Add +1 to the result if a target is crossing the
area. The suppressed zone lasts till the \emph{Characters} \emph{Initiative Score}.

\emph{Suppression Fire} can not be used if the\emph{Required Strength} of the
weapon is not met by the \emph{Character}.

\subsubsection{Melee Actions}
\index{Melee Action}

\myparagraph{Advance}
\index{Advance}

\begin{tabular}{rl}
    \textbf{Action} & Full                    \\
    \textbf{Test}   & \emph{Melee Weapons} +3 \\
    \textbf{Damage} & Optional                \\
\end{tabular}

\hfill

The target of this \emph{Melee Attack} must retreat 1 meter for every two points
of \emph{Test Quality}, or be hit with the \emph{Test Quality}. If the target
choses to retreat, the \emph{Character} \emph{advances} the same number of meters.

\myparagraph{Charge}
\index{Charge}

\begin{tabular}{rl}
    \textbf{Action} & Full                    \\
    \textbf{Test}   & \emph{Melee Weapons} -2 \\
    \textbf{Damage} & Yes                     \\
\end{tabular}

\hfill

This \emph{Action} requires the \emph{Character} to run in a straight line to the
target using their \emph{Run Rate}. If the Melee Attack is successful the
\emph{Character} can add 1 point for every 2 meters crossed to the weapon damage.


\myparagraph{Disarm}
\index{Disarm Action}

\begin{tabular}{rl}
    \textbf{Action} & Full                       \\
    \textbf{Test}   & \emph{Melee Weapons} -3 -X \\
    \textbf{Damage} & No                         \\
\end{tabular}

\hfill

If this \emph{Melee Attack} is successful, an additional opposed
\emph{Strength Test} is performed with a \emph{Modifier} of X for
the \emph{Character}. If the \emph{Character} wins this \emph{Test}, the target
is disarmed.

\myparagraph{Disengage}
\index{Disengage}

\begin{tabular}{rl}
    \textbf{Action} & Full                 \\
    \textbf{Test}   & \emph{Melee Weapons} \\
    \textbf{Damage} & No                   \\
\end{tabular}

\hfill

If this \emph{Action} is successful, the \emph{Character} can move a number of
meters equal to their \emph{Walking Rate} without risking a \emph{Free Strike}
from the target.

\myparagraph{Feint}
\index{Feint}

\begin{tabular}{rl}
    \textbf{Action} & Full                    \\
    \textbf{Test}   & \emph{Melee Weapons} +3 \\
    \textbf{Damage} & No                      \\
\end{tabular}

\hfill

If this \emph{Action} is successful, the \emph{Character} receives a +3
\emph{Modifier} for the next \emph{Melee Action} against the target.

\myparagraph{Fast Melee Attack}
\index{Fast Melee Attack}

\begin{tabular}{rl}
    \textbf{Action} & Half                    \\
    \textbf{Test}   & \emph{Melee Weapons} -3 \\
    \textbf{Damage} & Yes                     \\
\end{tabular}

\hfill

The \emph{Character} is making a \emph{Melee Attack}.

\myparagraph{Melee Attack}
\index{Melee Attack}

\begin{tabular}{rl}
    \textbf{Action} & Full                 \\
    \textbf{Test}   & \emph{Melee Weapons} \\
    \textbf{Damage} & Yes                  \\
\end{tabular}

\hfill

The \emph{Character} is making a \emph{Melee Attack}.

\myparagraph{Multi Melee Attack}
\index{Multi Melee Attack}

\begin{tabular}{rl}
    \textbf{Action} & Full                                   \\
    \textbf{Test}   & \emph{Melee Weapons} -2/-1 per attack, \\
                    & -1 per additional target               \\
    \textbf{Damage} & Yes                                    \\
\end{tabular}

\hfill

The \emph{Character} is performing multiple \emph{Melee Attacks} against one or
more targets. Only one \emph{Test} is rolled by the \emph{Character},
\emph{modified} by -2 per attack and -1 per additional target.
This value is opposed by all targets individually.

Wielding two melee weapons decreases the attack \emph{Modifier} to -1 for
attacking a single target multiple times, while a double handed weapon decreases
the attack \emph{Modifier} to -1 for attacking multiple targets once (still
adding  -1 for each additional target).

\myparagraph{Power Attack}
\index{Power Attack}

\begin{tabular}{rl}
    \textbf{Action} & Full                    \\
    \textbf{Test}   & \emph{Melee Weapons} -2 \\
    \textbf{Damage} & Yes                     \\
\end{tabular}

\hfill

The target gets an additional negative \emph{Modifier} of twice the
\emph{Strength} difference between the \emph{Character} and the target.

\myparagraph{Precise Strike}
\index{Precise Strike}

\begin{tabular}{rl}
    \textbf{Action} & Full                    \\
    \textbf{Test}   & \emph{Melee Weapons} -2 \\
    \textbf{Damage} & Yes                     \\
\end{tabular}

\hfill

If successful, this \emph{Action} adds 2X to the \emph{Melee Attack Test Quality}.

\subsection{Reactions}\index{Reaction}

In contrast to \emph{Actions}, \emph{Reactions} are \emph{declared} as a reaction
towards an \emph{Action} of another \emph{Character} or an event in the game world.
No \emph{Interrupt} is required to \emph{declare} them. Most \emph{Reactions} define
precisely how they interact with the triggering \emph{Action}.

Apart from this, \emph{Reactions} are normal \emph{Actions} with all rules and
limitations, especially the \emph{Maximum Initiative Score}.

\subsubsection{General Reactions}

\myparagraph{Dodge}
\begin{tabular}{rl}
    \textbf{Action} & Half/Simple/Free \\
    \textbf{Test}   & Dodge-2/-3/-4    \\
\end{tabular}

\hfill

The \emph{Character} is dodging an incoming \emph{Attack Action} that the
\emph{Character} is aware of, turning the attack into an \emph{opposed Test}.

The \emph{Character} can chose to drop to the ground while \emph{dodging},
adding a +2 \emph{Modifier} but also being \emph{disoriented} afterwards.

If the \emph{Character} is not dropping down, they may chose to move 1 meter
for for every 3 points of \emph{Test Quality}.

\subsubsection{Melee Reactions}

\myparagraph{Disarm}
\index{Disarm Reaction}

\begin{tabular}{rl}
    \textbf{Action} & Full                       \\
    \textbf{Test}   & \emph{Melee Weapons} -3 -X \\
    \textbf{Damage} & No                         \\
\end{tabular}

\hfill

The \emph{Character} is opposing a \emph{Melee Attack}. If this \emph{Reaction}
is successful, not only is the attack avoided, but an
additional opposed \emph{Strength Test} is performed with a \emph{Modifier}
of X for the \emph{Character}. If the \emph{Character} wins this \emph{Test},
the target is disarmed.

\myparagraph{Free Strike}
\index{Free Strike}

\begin{tabular}{rl}
    \textbf{Action} & Half                 \\
    \textbf{Test}   & \emph{Melee Weapons} \\
    \textbf{Damage} & Yes                  \\
\end{tabular}

\hfill

The \emph{Character} performs a \emph{Melee Attack} against a target that is
moving out of their \emph{Control Area}.


\myparagraph{Masterful Parry}
\index{Masterful Parry}

\begin{tabular}{rl}
    \textbf{Action} & Half                       \\
    \textbf{Test}   & \emph{Melee Weapons} -1 -X \\
    \textbf{Damage} & No                         \\
\end{tabular}

\hfill

The \emph{Character} is opposing a \emph{Melee Attack}. If this \emph{Action} is
successful, the \emph{Character} is getting a +X \emph{Modifier} for their next
\emph{Action} against the attacker.

\myparagraph{Parry}
\index{Parry}

\begin{tabular}{rl}
    \textbf{Action} & Half/Simple/Free             \\
    \textbf{Test}   & \emph{Melee Weapons} 0/-1/-2 \\
    \textbf{Damage} & No                           \\
\end{tabular}

\hfill

The \emph{Character} is opposing a \emph{Melee Attack}.

\myparagraph{Riposte}
\index{Riposte}

\begin{tabular}{rl}
    \textbf{Action} & Full                    \\
    \textbf{Test}   & \emph{Melee Weapons} -3 \\
    \textbf{Damage} & No                      \\
\end{tabular}

\hfill

The \emph{Character} is opposing a \emph{Melee Attack}. If the \emph{Riposte} is
successful the attacker is themselves hit and needs to resolve a
hit with the \emph{Test Quality} of the \emph{Riposte Action}.

\section{Attack Resolution}

\subsection{Ranged Combat}

\subsection{Melee Combat}

\section{Hit Resolution}

\subsection{Hit Location}

\subsection{Damage}