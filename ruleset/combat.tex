\chapter{Combat}

\section{Timing}

\subsection{Resolution Order}

In Pink Trenchcoat combat is resolved by continuous, always increasing time value.

\paragraph{Tick}
\index{Tick}
This time value is measured in \emph{Ticks}. A \emph{Tick} is
a time measure of about 0.3 seconds.

\paragraph{Current Tick}
\index{Current Tick}
Combat continuously advances the \emph{Current Tick} that represents the the current
point in time.

\paragraph{Initiative Score}
\index{Initiative Score}
The\ emph{Initiative Score} represents the \emph{Current Tick} in which a
\emph{Character} can \emph{declare} and take \emph{Actions}.
Ties in \emph{Initiative Score} are broken by the \emph{Characters Reflex} value.

\paragraph{Phase}
\index{Phase}
A \emph{Phase} lasts 20 \emph{Ticks}. Each Phase, this means on \emph{Current Tick }
20,40,80 and so on \emph{Continuous Effects} like fire,
toxin damage and bleeding are resolved.

\paragraph{Combat Flow}
\index{Declare Action}
\index{Resolve Action}
Characters who's \emph{Initiative Score} matches the \emph{Current Tick}
are allowed to \emph{declare} an \emph{Action}. An \emph{Action} that would
increase a \emph{Character's Initiative Score} to greater than the

\begin{equation}
    \textit{Maximum Initiative Score} = \textit{Current Tick} + 20
\end{equation}

can not be declared.

After \emph{declaring} an \emph{Action}, if no \emph{Interrupts} occur, the
\emph{Action} is \emph{resolved}. Finally, once the \emph{Character} has taken an
\emph{Action}, their \emph{Initiative Score} is increased by a value
depending on the \emph{Action}.

When all eligible \emph{Characters} have taken their \emph{Actions}, the
\emph{Current Tick} is advanced to the next meaningful value which is normally
either the next lowest \emph{Initiative Score} of a \emph{Character}, the next
\emph{Phase} if there are any \emph{Continuous Effects} to \emph{resolve} or the
\emph{Tick} a \emph{Character} in \emph{Delay} wants to \emph{act}.


\paragraph{Interrupt}
Instead or in addition to acting on their \emph{Initiative Score}, a \emph{Character}
can also chose to \emph{interrupt} another \emph{Character} in their
\emph{Initiative Score}. After a \emph{Character} declared their \emph{Action},
but before it was \emph{resolved}, the \emph{interrupting Character} can 
\emph{declare} their \emph{interrupting Action}.
The \emph{Interrupting Action} can itself be interrupted by a
\emph{Character} that has not yet declared an \emph{Action} this Tick.

A \emph{Reflex Test} decides the order in which \emph{Characters} taking part in
the \emph{Interrupt} are \emph{resolving} their \emph{Actions}. Each \emph{Character} 
receives a \emph{Modifier} equal to:

\begin{equation}
    \textit{Interrupt Modifier} = \textit{Current Tick} - \textit{Initiative Score}
\end{equation}

\emph{Actions} are \emph{resolved} starting from the highest \emph{Test Quality} 
to the lowest. Ties are broken first by lowest \emph{Initiative Score}, then by 
\emph{Reflex} value.


\subsection{Starting Combat}

\paragraph{Reflex}

\paragraph{Surprise}

\subsection{Combat Actions}

\subsubsection{Action Times}

\paragraph{Free Action}

\paragraph{Simple Action}

\paragraph{Half Action}

\paragraph{Full Action}

\subsubsection{Non-Actions}

\paragraph{Delay}

\paragraph{Move}

\paragraph{Overwatch}

\subsubsection{Continuous Inclusie Actions}

\paragraph{Leadership}

\paragraph{Tactics}

\subsubsection{General Actions}

\paragraph{Change Facing}

\paragraph{Change Posture}

\paragraph{Change Stance}

\paragraph{Extinguish Fire}

\paragraph{Ready Weapon}

\paragraph{Reload Weapon}

\paragraph{Sprint}

\paragraph{Talk}

\paragraph{Zigzag}

\subsubsection{Ranged Actions}

\paragraph{Aim}

\paragraph{Brace Weapon}

\paragraph{Fast Ranged Attack}

\paragraph{Burst Ranged Attack}

\paragraph{Multi Ranged Attack}

\paragraph{Ranged Attack}

\paragraph{Suppression Fire}

\subsubsection{Melee Actions}

\paragraph{Advance}

\paragraph{Charge}

\paragraph{Disarm}

\paragraph{Disengange}

\paragraph{Feint}

\paragraph{Fast Melee Attack}

\paragraph{Melee Attack}

\paragraph{Multi Melee Attack}

\paragraph{Power Attack}

\paragraph{Precise Strike}

\subsubsection{General Reactions}

\paragraph{Dodge}

\subsubsection{Melee Reactions}

\paragraph{Disarm}

\paragraph{Free Strike}

\paragraph{Masterful Parry}

\paragraph{Parry}

\paragraph{Riposte}

\section{Hit Resolution}

\subsection{Hit Location}

\subsection{Damage}