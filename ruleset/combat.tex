\chapter{Combat}

\section{Timing}

\subsection{Resolution Order}

In Pink Trenchcoat combat is resolved by continuous, always increasing time value.

\paragraph{Tick}
\index{Tick}
This time value is measured in \emph{Ticks}. A \emph{Tick} is
a time measure of about 0.3 seconds.

\paragraph{Current Tick}
\index{Current Tick}
Combat continuously advances the \emph{Current Tick} that represents the the current
point in time.

\paragraph{Initiative Score}
\index{Initiative Score}
The\ emph{Initiative Score} represents the \emph{Current Tick} in which a
\emph{Character} can \emph{declare} and take \emph{Actions}.
Ties in \emph{Initiative Score} are broken by the \emph{Characters Reflex} value.

\paragraph{Phase}
\index{Phase}
A \emph{Phase} lasts 20 \emph{Ticks}. Each Phase, this means on \emph{Current Tick }
20,40,80 and so on \emph{Continuous Effects} like fire,
toxin damage and bleeding are resolved.

\paragraph{Combat Flow}
\index{Declare Action}
\index{Resolve Action}
\index{Maximum Initiative Score}
\label{par:maximum initiative score}
Characters who's \emph{Initiative Score} matches the \emph{Current Tick}
are allowed to \emph{declare} an \emph{Action}. An \emph{Action} that would
increase a \emph{Character's Initiative Score} to greater than the

\begin{equation}
    \textit{Maximum Initiative Score} = \textit{Current Tick} + 20
\end{equation}

can not be declared.

After \emph{declaring} an \emph{Action} the \emph{Characters} \emph{Initiative Score}
is immediately increased by a value depending on the \emph{Action}. If no
\emph{Interrupts} occur, the \emph{Action} is \emph{resolved}.

When all eligible \emph{Characters} have taken their \emph{Actions}, the
\emph{Current Tick} is advanced to the next meaningful value which is normally
either the next lowest \emph{Initiative Score} of a \emph{Character}, the next
\emph{Phase} if there are any \emph{Continuous Effects} to \emph{resolve} or the
\emph{Tick} a \emph{Character} in \emph{Delay} wants to \emph{act}.


\paragraph{Interrupt}
Instead or in addition to acting on their \emph{Initiative Score}, a \emph{Character}
can also chose to \emph{interrupt} another \emph{Character} after \emph{declaring} an
\emph{Action}. After a \emph{Character} declared their \emph{Action} and
increased their \emph{Initiative Score},
but before it was \emph{resolved}, the \emph{interrupting Character} can
\emph{declare} their \emph{interrupting Action} (and also immediately increase their
\emph{Initiative Score}).
The \emph{Interrupting Action} can itself be interrupted by a
\emph{Character} that has not yet declared an \emph{Action} this \emph{Tick}.

A \emph{Reflex Test} decides the order in which \emph{Characters} taking part in
the \emph{Interrupt} are \emph{resolving} their \emph{Actions}. Each \emph{Character}
receives a \emph{Modifier} equal to:

\begin{equation}
    \textit{Interrupt Modifier} = \textit{Current Tick} - \textit{Initiative Score}
\end{equation}

In addition, the following \emph{Modifiers} apply:
\begin{table}[htb]
    \rowcolors{1}{}{lightgray}
    \caption[Interrupt Modifiers]{Interrupt Modifiers}
    \label{tab:interrupt modifiers}
    \centering
    \begin{tabular}{cl}
        \toprule
        \textbf{Modifier} & \textbf{Situation}                                \\
        \midrule
        \textbf{+9}       & \emph{Overwatch Action} triggered                 \\
        \textbf{+3}       & \emph{Aiming} or \emph{Watching} \emph{Character} \\
        \textbf{-3}       & per level of \emph{Interruption}                  \\
        \bottomrule
    \end{tabular}
\end{table}

\emph{Actions} are \emph{resolved} starting from the highest \emph{Test Quality}
to the lowest. Ties are broken first by lowest \emph{Initiative Score}, then by
\emph{Reflex} value.

\subsection{Starting Combat}

Once the Game Master decides that the Game should transition from
\emph{Narrative Time} to \emph{Combat Time}, the \emph{Characters} starting
\emph{Initiative Score} needs to be determined. To do so, each Character rolls
a \emph{Reflex Test}. The negative \emph{Test Quality} determines the initial
\emph{Initiative Score}.

Apply the following \emph{Modifiers}:

\begin{table}[htb]
    \rowcolors{1}{}{lightgray}
    \caption[Surprise Modifiers]{Surprise Modifiers}
    \label{tab:surprise modifiers}
    \centering
    \begin{tabular}{cl}
        \toprule
        \textbf{Modifier} & \textbf{Character State}        \\
        \midrule
        \textbf{+6}       & Initiated first \textit{Action} \\
        \textbf{0}        & Actively anticipating combat    \\
        \textbf{-3}       & Suspicious                      \\
        \textbf{-6}       & Not expecting combat            \\
        \textbf{-9}       & Deeply involved                 \\
        \bottomrule
    \end{tabular}
\end{table}

\section{Combat Actions}

\subsection{Action Times}
To reduce cognitive load, \emph{Action Times} are separated into four categories.
Depending
on the \emph{Character} and their \emph{Attribute}values, these \emph{Actions} take
a different amount of \emph{Ticks} to fulfil.

\paragraph{Free Action}
\index{Free Action}
\emph{Free Actions} are the shortest kind of \emph{Action}. They require no or almost no thoughts
and can be executed almost immediately.

\paragraph{Simple Action}
\index{Simple Action}
A \emph{Simple Action} can still be done quickly, or be triggered by reflex, but is
not instant.

\paragraph{Half Action}
\index{Half Action}
\emph{Half Actions }are the standard basic Action.

\paragraph{Full Action}
\index{Full Action}
\emph{Actions} taking a lot of concentration or a lot of time.

\subsection{Meta-Actions}

Meta Actions are different compared to normal Actions in a sense that they do not
follow the typical scheme of Declaration, Initiative Score increase and Resolution.

\paragraph{Delay}
\index{Delay}
A \emph{delaying} Character does not have a \emph{Initiative Score}, but can chose to
\emph{act} at any future \emph{Current Tick} and thus immediately gets the
\emph{Current Tick} assigned as \emph{Initiative Score}, before \emph{declaring }
the \emph{Action}. They have to do so before
any other \emph{Character} \emph{declared} an \emph{Action} in that \emph{Tick}.
If another \emph{Character} already \emph{declared} an \emph{Action},
the \emph{delaying} \emph{Character} can still chose to \emph{act},
but will need to perform an \emph{Interrupt}.

After performing any \emph{Action}, \emph{delay} ends.

\paragraph{Watch}
\index{Watch}
When \emph{declaring Watch} a \emph{Character} has to chose a suitable
\emph{Character} or object to \emph{watch}. The \emph{Character} immediately loses
their \emph{Initiative Score}. The watching Character is granted a bonus of

A \emph{Character} can switch from \emph{Watch} to \emph{Delay} at any time.

\paragraph{Overwatch}
\index{Overatch}
When \emph{declaring Overwatch} a \emph{Character} has to chose both a
\emph{specific Condition} and a \emph{specific Action}. The \emph{Character}
immediately loses their \emph{Initiative Score}.

Once the \emph{specific Condition} is fulfilled, even if in \emph{declaration} the
\emph{Character} immediately performs the \emph{specific Action}. The \emph{Character}
is assigned an \emph{Initiative Score} according to the \emph{Current Tick}
plus the value depending on the \emph{Action} and \emph{Overwatch} ends.
If \emph{specific Condition} was declared by another \emph{Character} an
\emph{Interrupt} is triggered. The \emph{Character} on \emph{Overwatch} receives
a +9 \emph{Modifier} for the \emph{Reflex Test} in this case.

\emph{Specific Conditions} should not be too complex and should be easily
identified as true by the \emph{Character} on \emph{Overwatch} with the current
situation.

\emph{Specific Actions} can only be \emph{Active Actions}.

A \emph{Character} can switch from \emph{Overwatch} to \emph{Delay} at any time.

\paragraph{Move}
\index{Move Action}
Normal movement, like crawling, walking or running, can be performed in addition
to \emph{Active Actions}. However, if of another \emph{Action} is performed while
moving, a \emph{Movement Modifier} is applied.

The \emph{Movement Modifier}, as well as the distance moved depend both on the
movement style as well as the duration of the movement, which is represent by the
\emph{Action Time}.

A \emph{Move Action} can only performed in addition to an\emph{ Active Action}
if the \emph{Action} does not include movement.
\begin{table}[htb]
    \rowcolors{1}{}{lightgray}
    \caption[Movement Modifiers]{Movement Modifiers}
    \label{tab:movement modifiers}
    \centering
    \begin{tabular}{cl}
        \toprule
        \textbf{Modifier} & \textbf{Movement Style} \\
        \midrule
        \textbf{-6}       & Crawl                   \\
        \textbf{-1}       & Walk                    \\
        \textbf{-3}       & Run                     \\
        \bottomrule
    \end{tabular}
\end{table}

\paragraph{Ongoing Actions}
\index{Onging Action}

\emph{Ongoing Actions} represent \emph{Actions} that take, longer than 20
\emph{Ticks}, sometimes considerably longer. They are represented by increasing
the \emph{Initiative Score} of the \emph{Character} performing the \emph{Action}
by 20 if the \emph{Character} wants to continue doing the \emph{Ongoing Action},
till it is finished.

\subsection{Continuous Inclusive Actions}

\paragraph{Leadership}

\paragraph{Tactics}

\subsection{Active Actions}
\index{Active Action}

\subsubsection{General Actions}


\paragraph{Free Actions}
Free Actions without specific rules include:
\begin{itemize}[parsep=0em]
    \item dropping an objects
\end{itemize}

\paragraph{Simple Actions}
Simple Actions without specific rules include:
\begin{itemize}[parsep=0em]
    \item drop to the ground
\end{itemize}

\paragraph{Half Actions}
Half Actions without specific rules include:
\begin{itemize}[parsep=0em]
    \item dropping an objects
\end{itemize}

\paragraph{Full Actions}
Full Actions without specific rules include:
\begin{itemize}[parsep=0em]
    \item perform a \emph{Perception Test}
\end{itemize}


\paragraph{Change Facing}

\paragraph{Change Posture}

\paragraph{Change Stance}

\paragraph{Extinguish Fire}

\paragraph{Ready Weapon}

\paragraph{Reload Weapon}

\paragraph{Sprint}

\paragraph{Talk}

\paragraph{Zigzag}

\subsubsection{Ranged Actions}

\paragraph{Aim}

\paragraph{Brace Weapon}

\paragraph{Fast Ranged Attack}

\paragraph{Burst Ranged Attack}

\paragraph{Multi Ranged Attack}

\paragraph{Ranged Attack}

\paragraph{Suppression Fire}

\subsubsection{Melee Actions}

\paragraph{Advance}

\paragraph{Charge}

\paragraph{Disarm}

\paragraph{Disengange}

\paragraph{Feint}

\paragraph{Fast Melee Attack}

\paragraph{Melee Attack}

\paragraph{Multi Melee Attack}

\paragraph{Power Attack}

\paragraph{Precise Strike}

\subsection{Reactions}
\index{Reaction}

In contrast to \emph{Actions}, \emph{Reactions} are \emph{declared} as a reaction
towards an \emph{Action} of another \emph{Character} or an event in the game world.
No \emph{Interrupt} is required to \emph{declare} them. Most \emph{Reactions} define
precisely how they interact with the triggering \emph{Action}.

Apart from this, \emph{Reactions} are normal \emph{Actions} with all rules and
limitations, especially the \emph{Maximum Initiative Score}.

\subsubsection{General Reactions}

\paragraph{Dodge}

\subsubsection{Melee Reactions}

\paragraph{Disarm}

\paragraph{Free Strike}

\paragraph{Masterful Parry}

\paragraph{Parry}

\paragraph{Riposte}

\section{Hit Resolution}

\subsection{Hit Location}

\subsection{Damage}