\chapter{Computers}
\label{chap:basics}

This chapter explains both the matrix, including AR and everything computer related
like electronic warfare.


\section{What is the Matrix}

The Matrix is a virtual representation of the cyberspace for human users. It is they way they
perceive interactions between themselves and both other matrix users and \emph{Matrix Entities}.

\subsection{Accessing the Matrix}
\label{subsec:access matrix}

There are various ways to access the matrix.

\paragraph{Physical Access}
This method of matrix access uses outdated methods like keyboard and mouse.
It is generally outdated and very slow. It is only used if people are afraid of any kind of matrix
damage, or are very traditional.

\paragraph{Augmented Reality}
Augmented Reality or AR access is a widely used for of matrix access, especially one the go
or while wanting to do things in parallel. AR users still see the real world, but get additional
information projected on top of it. Thus they can see objects, additional information and also
sound added to the real world that does not exist.

\paragraph{Virtual Reality}
Virtual Reality supersedes the perception of the user. They are not aware of the real world, but
instead see, hear, smell and feal virtual sensory input that is 100\% artificial.

\subparagraph{Tortoise}
Tortoise uses not direct brain interfaces as provided by most data jacks, but uses
outdated technologies like trodes. Due to it not requiring cyberware it is often
used by adepts or magicians.

\subparagraph{Cold Sim}
Cold Sim is the standard way of using the matrix today. The user is experiencing
the matrix by direct stimulation of their sensory cortex so that they see, hear
and feel the matrix. Their thoughts of movements and actions are translated into
commands of their virtual bodies using virtual applications.

\subparagraph{Hot Sim}
Hot Sim is the most dangerous but also the fastest way to access the matrix.
The data is directly fed into the users brain even circumventing their sensory
centers that are stimulated in cold sim. Instead, using knowledge link technology,
the matrix user just instantly knows the information. Also their raw thoughts
are transformed into matrix commands.


\begin{table}[htb]
    \caption[Matrix Access Methods]{Matrix Access Methods}
    \label{tab:matrix access}
    \centering
    \begin{tabular}{cll}
        \toprule
        \textbf{Method}   & \textbf{Input}       & \textbf{Output}         \\
        \midrule
        \textbf{Physical} &
        \makecell[l] {\tabitem Keyboard                                    \\ \tabitem Mouse \\ \tabitem Touchscreen \\ \tabitem Input Trigger} &
        \makecell[l] {\tabitem Screen                                      \\ \tabitem Loudspeaker}        \\
        \midrule
        \textbf{AR}       &
        \makecell[l] {\tabitem Transducer                                  \\ \tabitem Microphone \\ \tabitem AR Gloves \\ \tabitem Holo Scanner} &
        \makecell[l] {\tabitem Lenses                                      \\ \tabitem Vision-Link \\ \tabitem In-Ears \\ \tabitem Sound-Link}        \\
        \midrule
        \textbf{Tortoise} &
        \makecell[l] {\tabitem Trodes                                      \\ \tabitem External \\ Sim Rig} &
        \makecell[l] {\tabitem Trodes                                      \\ \tabitem External \\ Sim Module}        \\
        \midrule
        \textbf{Cold Sim} & \tabitem Sim Rig     & \tabitem Sim Module     \\
        \midrule
        \textbf{Hot Sim}  & \tabitem Transcriber & \tabitem Knowledge Link \\
        \bottomrule
    \end{tabular}
\end{table}

\begin{table}[htb]
    \rowcolors{1}{}{lightgray}
    \caption[Matrix Access Requirements]{Matrix Access Requirements}
    \label{tab:matrix access requirements}
    \centering
    \begin{tabular}{lc}
        \toprule
        \textbf{Method}   & \textbf{Processor}/ \textbf{Uplink} \\
        \midrule
        \textbf{Physical} & 1                                   \\
        \textbf{AR}       & 3                                   \\
        \textbf{Tortoise} & 6                                   \\
        \textbf{Cold Sim} & 6                                   \\
        \textbf{Hot Sim}  & 10                                  \\
        \bottomrule
    \end{tabular}
\end{table}

\begin{table}[htb]
    \rowcolors{1}{}{lightgray}
    \caption[Matrix Access Modifiers]{Matrix Access Modifiers}
    \label{tab:matrix access mods}
    \centering
    \begin{tabular}{lcccc}
        \toprule
        \textbf{Method}   & \textbf{Skill} & \textbf{React} & \textbf{Tick} & \textbf{Damage} \\
        \midrule
        \textbf{Physical} & -3             & -5             & x6            & None            \\
        \textbf{AR}       & -2             & -3             & x3            & Fatigue         \\
        \textbf{Tortoise} & -1             & -2             & x1.5          & Fatigue         \\
        \textbf{Cold Sim} & 0              & 0              & x1            & Stun            \\
        \textbf{Hot Sim}  & +2             & +3             & x0.7          & Physical        \\
        \bottomrule
    \end{tabular}
\end{table}

\section{Matrix building blocks}


\subsection{Matrix Devices}

The Matrix is made up of hardware that is processing and delivering it. Most notable are are
the different pieces of hardware the matrix is running on. In general four different classes of
matrix hardware can be found.

\paragraph{Gadget}
\index{Gadget}
Gadgets are small and cheap pieces of hardware. Some of them are so cheap, they can be found in
throwaway articles like food packaging. Others are powering small sensors or track positions. They
range from pinhead size to coin size.
A typical person is carrying around dozens of them.

\paragraph{Commlink}
\index{Commlink}
Commlinks are not only the most common mans to communicate but also a matrix hardware class.
They are bigger than gadgets, but the smallest of them can fit into a bigger earring. The
standard size is of an average playing card. They carry enough processing power to allow for at
least \emph{Augmented Reality}.

\paragraph{Cyberdeck}
\index{Cyberdeck}
Cyberdecks are a special form factor that only few people need. Much bigger than a an average
commlink, about the size of a shoe-box, they pack much more processing power. Most cyberdecks are
used for illegal purposes and are equipped with a \emph{Sleaze} module to avoid detection in the
matrix.

\paragraph{Mainframe}
\index{Mainframe}
Mainframes are stationary pieces of matrix hardware. They range from shoe-box size to whole floors
of a building. Mainframes are used to service multiple people or perform high performance
computations.


\subsection{Matrix Entities}

Matrix entities are virtual building blocks of the matrix. Although they have a physical basis,
they are purely virtual representations both in virtual- and augmented reality.

\paragraph{Node}
\index{Node}
\label{par:node}

A Node is a matrix entity with processing power. It has matrix location and can be
\emph{accessed}. A Node can run \emph{Processes}, store \emph{Files} and be the origin or
destination of a \emph{Stream}.

\paragraph{Process}
\index{Process}
\label{par:process}

Processes are matrix entities that actively perform actions. They are running on their origin
\emph{Node}.

\subparagraph{Persona}
\index{Persona}
A Persona is a special kind of \emph{Process} that represents a matrix user and their actions.

\subparagraph{Program}
\index{Program}
A program is a piece of software that can be used by a \emph{Persona} or an \emph{Agent} as a tool
to perform various actions. Programs are always attached to a \emph{Persona} or \emph{Agent}.

\subparagraph{Agent}
\index{Agent}
An agent is a process that can perform autonomous decisions and use \emph{Programs} to perform
actions. Agents can \emph{access} \emph{Nodes}.

\subparagraph{ICE}
\index{ICE}
ICE, or Intrusion Countermeasures, are \emph{Agents} with the special purpose to defend a node
from hackers.

\paragraph{Streams}
\index{Stream}
\label{par:stream}

A stream connects two \emph{Nodes}, the origin and the destination, with a data connection.
A stream also connects the \emph{Node} a \emph{Persona} or \emph{Agent} is running on with
the \emph{Node} it is \emph{accessing}.

\paragraph{File}
\index{File}
\label{par:file}

\subsection{Matrix Attributes}

Each \emph{matrix device} has a number of attributes that define its properties in the matrix.

\paragraph{Processor}
\index{Processor}

The Processor attribute represents a \emph{Nodes} row computing power. As most
devices are very advanced, a high Processor rating is not needed for most every
day tasks. High Processor ratings are required for intensive tasks like processing
Sim-Sense signals for example when using \emph{Cold Sim} or the even more complex
\emph{Hot Sim}. The attribute is also useful if a mainframe is supporting a large
user base.

It is also important in matrix combat where combatants try to overwhelm
the opponents \emph{Node}.

\hfill

The Processor attribute is mostly related to a devices size. The bigger a device
the higher its rating is on average.

\begin{table}[htb]
    \rowcolors{1}{}{lightgray}
    \caption[Processor Ratings]{Processor Ratings}
    \label{tab:processor ratings}
    \centering
    \begin{tabular}{lc}
        \toprule
        \textbf{Entity}    & \textbf{Processor} \\
        \midrule
        \textbf{Gadget}    & 0-4                \\
        \textbf{Commlink}  & 3-8                \\
        \textbf{Cyberdeck} & 6-13               \\
        \textbf{Mainframe} & 8-21               \\
        \bottomrule
    \end{tabular}
\end{table}


\paragraph{System}
\index{System}

System describes the quality of the operating system and standard software
suite of a \emph{Node}. The higher the ranking the higher the rating of
\emph{Programs} that can be run.

A high Systems rating also helps autonomous software like \emph{ICE} to
perform more efficiently.

\paragraph{Firewall}
\index{Firewall}

Firewall represents the resilience of a \emph{Node} against anything
illegal. This includes any kind of \emph{Exploit} actions leading to
illegal actions not governed by the users level.

Firewall is not determined by a \emph{Nodes} computing power but by
the skill and time invested by the maintainers of the node, and the
number of users and different \emph{Processes} it is supporting.

\hfill

Firewall Ratings are often given by a color coding.

\begin{table}[htb]
    \rowcolors{1}{}{lightgray}
    \caption[Firewall Ratings]{Firewall Ratings}
    \label{tab:firwall ratings}
    \centering
    \begin{tabular}{lc}
        \toprule
        \textbf{Color}        & \textbf{Firewall} \\
        \midrule
        \textbf{Blue}         & 0-4               \\
        \textbf{Green}        & 5-9               \\
        \textbf{Orange}       & 10-14             \\
        \textbf{Red}          & 15-19             \\
        \textbf{Ultra Violet} & 20-21             \\
        \bottomrule
    \end{tabular}
\end{table}

\subparagraph{Blue} Blue \emph{Nodes} represent the lowest level of security.
They are often either very cheap gadgets like Smart Tags or public mainframes
like public libraries.

\subparagraph{Green} Green \emph{Nodes} represent the vast majority of matrix hosts.
They are a good trade-off between expensive security experts and time invest.
\emph{Nodes} with fewer users tend to have higher green ratings.

\subparagraph{Orange} Orange \emph{Nodes} are used when higher security is required,
like in the mainframe of a police station, a law firm, or the \emph{Nodes} of upper
class individuals.

\subparagraph{Red} Red \emph{Nodes} are mostly used by high security facilities like
corporate research sites or  government agencies.

\subparagraph{Ultra Violet} Ultra Violet \emph{Nodes}, if they exist, are only used
for legendary and top-secret institutions.


\paragraph{Uplink}
\index{Uplink}

Uplink describes the quality, speed and volume of data that a \emph{Node}
can access per time. A high throughput is required for \emph{Cold Sim} and even
more for \emph{Hot Sim}. Uplink mostly degrades over distance, although not as
fast as wireless \emph{Signal} does, or if the signal has to go through wireless
channels.

\paragraph{Signal}
\index{Signal}

The Signal rating describes the power and quality of a wireless signal. It is
used to check how far a signal penetrates and also represents the power delivered
in case of \emph{Electronic Warfare}.
Only nodes with wireless capabilities have a Signal rating.

\begin{table}[htb]
    \rowcolors{1}{}{lightgray}
    \caption[Signal Ranges]{Signal Ranges}
    \label{tab:signal ranges}
    \centering
    \begin{tabular}{clcl}
        \toprule
        \textbf{Signal} & \textbf{Range} & \textbf{Signal} & \textbf{Range} \\
        \midrule
        \textbf{0}      & 1 m            & \textbf{11}     & 5 km           \\
        \textbf{1}      & 2 m            & \textbf{12}     & 10 km          \\
        \textbf{2}      & 5 m            & \textbf{13}     & 20 km          \\
        \textbf{3}      & 10 m           & \textbf{14}     & 50 km          \\
        \textbf{4}      & 20 m           & \textbf{15}     & 100 km         \\
        \textbf{5}      & 50 m           & \textbf{16}     & 200 km         \\
        \textbf{6}      & 100 m          & \textbf{17}     & 500 km         \\
        \textbf{7}      & 200 m          & \textbf{18}     & 1,000 km       \\
        \textbf{8}      & 500 m          & \textbf{19}     & 2,000 km       \\
        \textbf{9}      & 1 km           & \textbf{20}     & 5,000 km       \\
        \textbf{10}     & 2 km           & \textbf{21}     & 10,000 km      \\
        \bottomrule
    \end{tabular}
\end{table}

\paragraph{Sleaze}
\index{Sleaze}

Only devices equipped with with an illegal sleaze module have a Sleaze rating.
The Sleaze rating allows a decker to hide from security software of a \emph{Node}.
Without it the decker would instantly be recognized after performing any kind of
\emph{Exploit} action.

\section{Matrix Actions}

\begin{table*}[t]
    %\rowcolors{3}{}{lightgray}
    \caption[Matrix Actions]{Matrix Actions}
    \label{tab:matrix actions}
    \centering
    \begin{tabular}{ccllll}
        \toprule
        \textbf{Account Level} & \textbf{Program} & \textbf{Node}                                 & \textbf{Process}              & \textbf{Stream}               & \textbf{File}    \\
        \midrule
        \multirow{5}{*}[-2em]
        {\textbf{Anonymous}}   & None             & \tabitem Access                               & \tabitem Command              & \tabitem Read                 & \tabitem Read    \\
        \cmidrule{3-6}
                               & Analyze          & \tabitem Analyze                              & \tabitem Analyze              & \tabitem Analyze              & \tabitem Analyze \\
        \cmidrule{3-6}
                               & Break            &                                               &                               & \tabitem Break                & \tabitem Break   \\
        \cmidrule{3-6}
                               & Corrupt          & \makecell[l]{\tabitem Crash                                                                                                      \\ \tabitem Slow} & \makecell[l]{\tabitem Crash \\ \tabitem Slow} & \tabitem Corrupt & \tabitem Corrupt \\
        \cmidrule{3-6}
                               & Find             & \tabitem Find                                 & \tabitem Find                 & \tabitem Find                 & \tabitem Find    \\
        \cmidrule{2-6}

        \multirow{5}{*}[-2.5em]
        {\textbf{User}}        & None             & \tabitem User Account Access                  & \makecell[l]{\tabitem Command                                                    \\  \tabitem Start \\ \tabitem Stop} & \makecell[l]{\tabitem Read \\ \tabitem Start \\  \tabitem Send \\ \tabitem Terminate } & \makecell[l]{\tabitem Create \\ \tabitem Delete \\ \tabitem Read \\ \tabitem Write} \\
        \cmidrule{3-6}
                               & Control          &                                               & \tabitem Control [Thing]      &                               &                  \\
        \cmidrule{3-6}
                               & Crypt            &                                               &                               & \makecell[l]{\tabitem Decrypt                    \\ \tabitem Encrypt} & \makecell[l]{\tabitem Decrypt \\ \tabitem Encrypt} \\
        \cmidrule{3-6}
                               & Edit             &                                               &                               & \tabitem Edit                 & \tabitem Edit    \\
        \cmidrule{3-6}
                               & Medic            & \tabitem Repair                               & \tabitem Repair               &                               &                  \\
        \cmidrule{2-6}
        \textbf{Security}      & None             & \makecell[l]{\tabitem Security Account Access                                                                                    \\ \tabitem View Accounts \\  \tabitem View Alarm Status \\ \tabitem View Logs \\ \tabitem View Subscriptions } & \makecell[l]{\tabitem Command ICE \\ \tabitem Start ICE \\ \tabitem Stop ICE} & & \\
        \cmidrule{2-6}
        \textbf{Admin}         & None             & \makecell[l]{\tabitem Admin Account Access                                                                                       \\ \tabitem Change Alarm Status \\  \tabitem Edit Accounts \\ \tabitem Edit Logs \\ \tabitem Edit Subscriptions \\ \tabitem Shutdown  \\ \tabitem Startup} &  & &  \\
        \bottomrule
    \end{tabular}
\end{table*}

\paragraph{Access}
\label{par: access}

\mbox{}\\

\begin{tabular}{rl}
    \textbf{Program}       & None            \\
    \textbf{Prerequisite}  & \emph{Node} AID \\
    \textbf{Test Modifier} & None            \\
    \textbf{Duration}      & 0.1s            \\
\end{tabular}

\hfill

This action is required to access a \emph{Node} with a known AID. After a successful Access Action
the decker has accessed the \emph{Node}.

\paragraph{Analyze [Node, Process, Stream, File]}
\label{par: analyze}

\mbox{}\\

\begin{tabular}{rl}
    \textbf{Program}       & Analyze                             \\
    \textbf{Prerequisite}  & Found [Node, Process, Stream, File] \\
    \textbf{Test Modifier} & \emph{Sleaze}                       \\
    \textbf{Duration}      & 2s                                  \\
\end{tabular}

\hfill

This action allows for analyzing properties of various matrix entities. To analyze a \emph{Node}
an AID is required. Other entities have to be \emph{found}. \emph{Processes} and \emph{Streams}
can only be analyzed if the decker has \emph{accessed} either the target or the destination
\emph{Node}.

\begin{table}[htb]
    \rowcolors{1}{}{lightgray}
    \caption[Analyze Node Results]{Analyze Node Results}
    \label{tab:analyze results}
    \centering
    \begin{tabular}{cll}
        \toprule
        \textbf{Result} & \textbf{Properties}     & \textbf{Location} \\
        \midrule
        \textbf{0}      & Active Alert Status     &                   \\
        \textbf{2}      & AID                     &                   \\
        \textbf{4}      & Type                    &                   \\
        \textbf{6}      & High/Low Attributes     & Continent         \\
        \textbf{8}      & Functionality           & State             \\
        \textbf{10}     & High/Med/Low Attributes & City              \\
        \textbf{12}     & Active Processes        & Suburb            \\
        \textbf{14}     & Exact Attributes        & Street            \\
        \textbf{16}     &                         & Building          \\
        \textbf{18}     &                         & Room              \\
        \textbf{20}     &                         & Exact             \\
        \bottomrule
    \end{tabular}
\end{table}

\paragraph{Break}
\label{par: break}

\mbox{}\\

\begin{tabular}{rl}
    \textbf{Program}       & Break                             \\
    \textbf{Prerequisite}  & Found [\emph{File}/\emph{Stream}] \\
    \textbf{Test Modifier} & Crypt Rating +3                   \\
    \textbf{Duration}      & 20s                               \\
\end{tabular}

\hfill


\paragraph{Command}
\label{par: command}

\mbox{}\\

\begin{tabular}{rl}
    \textbf{Program}       & None               \\
    \textbf{Prerequisite}  & \emph{Process} AID \\
    \textbf{Test Modifier} & var.               \\
    \textbf{Duration}      & 2s                 \\
\end{tabular}

\hfill

This action allows a decker to give commands to a \emph{Process}. This can either be an agent,
or any other program on a \emph{Node} or \emph{Device} like a drone or a security camera.

The decker needs only the AID of the \emph{Process} and does not need to access the hosting
\emph{Node}.

\paragraph{Control}
\label{par: control}

\mbox{}\\

\begin{tabular}{rl}
    \textbf{Program}       & Control \\
    \textbf{Prerequisite}  & None    \\
    \textbf{Test Modifier} & None    \\
    \textbf{Duration}      & 1s      \\
\end{tabular}

\hfill

\paragraph{Corrupt}
\label{par: corrupt action}

\mbox{}\\

\begin{tabular}{rl}
    \textbf{Program}       & Corrupt                           \\
    \textbf{Prerequisite}  & Found [\emph{Stream}/\emph{File}] \\
    \textbf{Test Modifier} & Originating \emph{Node} System    \\
    \textbf{Duration}      & 1s                                \\
\end{tabular}

\hfill

\paragraph{Crash}
\label{par: crash}

\mbox{}\\

\begin{tabular}{rl}
    \textbf{Program}       & Corrupt                            \\
    \textbf{Prerequisite}  & Found [\emph{Node}/\emph{Process}] \\
    \textbf{Test Modifier} & System                             \\
    \textbf{Duration}      & 1s                                 \\
\end{tabular}

\hfill

\paragraph{Create File}
\label{par: create file}

\mbox{}\\

\begin{tabular}{rl}
    \textbf{Program}       & None \\
    \textbf{Prerequisite}  & None \\
    \textbf{Test Modifier} & None \\
    \textbf{Duration}      & 0.5s \\
\end{tabular}

\hfill

\paragraph{Decrypt}
\label{par: decrypt}

\mbox{}\\

\begin{tabular}{rl}
    \textbf{Program}       & Crypt \\
    \textbf{Prerequisite}  & None  \\
    \textbf{Test Modifier} & None  \\
    \textbf{Duration}      & 1s    \\
\end{tabular}

\hfill

\paragraph{Delete File}
\label{par: delete file}

\mbox{}\\

\begin{tabular}{rl}
    \textbf{Program}       & None \\
    \textbf{Prerequisite}  & None \\
    \textbf{Test Modifier} & None \\
    \textbf{Duration}      & 0.5s \\
\end{tabular}

\hfill

\paragraph{Encrypt}
\label{par: encrypt}

\mbox{}\\

\begin{tabular}{rl}
    \textbf{Program}       & Crypt \\
    \textbf{Prerequisite}  & None  \\
    \textbf{Test Modifier} & None  \\
    \textbf{Duration}      & 1s    \\
\end{tabular}

\hfill

\paragraph{Find Process}
\label{par: find process}

\mbox{}\\

\begin{tabular}{rl}
    \textbf{Program}       & Find                                     \\
    \textbf{Prerequisite}  & Access to origin/destination \emph{Node} \\
    \textbf{Test Modifier} & \emph{Sleaze}                            \\
    \textbf{Duration}      & 10s                                      \\
\end{tabular}

\hfill

This action allows a decker to find \emph{Processes} in a \emph{Node}, which must be either its
origin or the destination.

\paragraph{Find Stream}
\label{par: find stram}

\mbox{}\\

\begin{tabular}{rl}
    \textbf{Program}       & Find                                     \\
    \textbf{Prerequisite}  & Access to origin/destination \emph{Node} \\
    \textbf{Test Modifier} & var.                                     \\
    \textbf{Duration}      & 10s                                      \\
\end{tabular}

\hfill

This action allows a decker to find \emph{Streams} in a \emph{Node}, which must be either its
origin or the destination.

\paragraph{Find File}
\label{par: find file}

\mbox{}\\

\begin{tabular}{rl}
    \textbf{Program}       & Find                          \\
    \textbf{Prerequisite}  & Access to hosting \emph{Node} \\
    \textbf{Test Modifier} & var.                          \\
    \textbf{Duration}      & 10s                           \\
\end{tabular}

\hfill

This action allows a decker to find \emph{Files} in a \emph{Node}.

\paragraph{Read}
\label{par: read}

\mbox{}\\

\begin{tabular}{rl}
    \textbf{Program}       & None                            \\
    \textbf{Prerequisite}  & Found \emph{File}/\emph{Stream} \\
    \textbf{Test Modifier} & None                            \\
    \textbf{Duration}      & 0.1s                            \\
\end{tabular}

\hfill

This action allows a decker to read \emph{Files} in a \emph{Node}. The decker must have
\emph{found} the the \emph{File} first.

\paragraph{Repair}
\label{par: repair}

\mbox{}\\

\begin{tabular}{rl}
    \textbf{Program}       & Medic                           \\
    \textbf{Prerequisite}  & Found \emph{File}/\emph{Stream} \\
    \textbf{Test Modifier} & None                            \\
    \textbf{Duration}      & 0.1s                            \\
\end{tabular}

\hfill


\paragraph{Send to Stream}
\label{par: send to stream}

\mbox{}\\

\begin{tabular}{rl}
    \textbf{Program}       & None                \\
    \textbf{Prerequisite}  & Found \emph{Stream} \\
    \textbf{Test Modifier} & None                \\
    \textbf{Duration}      & 0.5s                \\
\end{tabular}

\hfill

\paragraph{Slow}
\label{par: slow}

\mbox{}\\

\begin{tabular}{rl}
    \textbf{Program}       & Corrupt                            \\
    \textbf{Prerequisite}  & Found [\emph{Node}/\emph{Process}] \\
    \textbf{Test Modifier} & System                             \\
    \textbf{Duration}      & 1s                                 \\
\end{tabular}

\hfill

\paragraph{Start Process}
\label{par: start process}

\mbox{}\\

\begin{tabular}{rl}
    \textbf{Program}       & None \\
    \textbf{Prerequisite}  & None \\
    \textbf{Test Modifier} & None \\
    \textbf{Duration}      & 0.5s \\
\end{tabular}

\hfill

\paragraph{Start Stream}
\label{par: start stream}

\mbox{}\\

\begin{tabular}{rl}
    \textbf{Program}       & None \\
    \textbf{Prerequisite}  & None \\
    \textbf{Test Modifier} & None \\
    \textbf{Duration}      & 0.5s \\
\end{tabular}

\hfill

\paragraph{Stop Process}
\label{par: stop process}

\mbox{}\\

\begin{tabular}{rl}
    \textbf{Program}       & None \\
    \textbf{Prerequisite}  & None \\
    \textbf{Test Modifier} & None \\
    \textbf{Duration}      & 0.5s \\
\end{tabular}

\hfill


\paragraph{Terminate Stream}
\label{par: terminate stream}

\mbox{}\\

\begin{tabular}{rl}
    \textbf{Program}       & None                \\
    \textbf{Prerequisite}  & Found \emph{Stream} \\
    \textbf{Test Modifier} & None                \\
    \textbf{Duration}      & 0.5s                \\
\end{tabular}

\hfill

\paragraph{Write to File}
\label{par: write file}

\mbox{}\\

\begin{tabular}{rl}
    \textbf{Program}       & None              \\
    \textbf{Prerequisite}  & Found \emph{File} \\
    \textbf{Test Modifier} & None              \\
    \textbf{Duration}      & 0.5s              \\
\end{tabular}

\hfill

\section{Cracking}
\label{sec: cracking}


\subsection{Exploit}
\index{Exploit}

Every time a decker wants to perform an action where their user level is not high
enough, like viewing the security log without being at least \emph{Security} level,
an Exploit test is required. If the action in question requires a test itself,
when for example editing a stream, the Exploit test does not replace the actual
test but is an additional requirement.

An Exploit test is an opposed test between the deckers \emph{Cracking(Exploit)} and
the \emph{Nodes} \emph{Firewall}.

\begin{verbatim}
    TQ = 10dF + Cracking(Exploit) - Firewall
\end{verbatim}

In addition each Exploit test can increase the deckers \emph{Security Tally}.


\begin{table}[htb]
    \rowcolors{3}{}{lightgray}
    \caption[Exploit Modifiers]{Exploit Modifiers}
    \label{tab:exploit mods}
    \centering
    \begin{tabular}{lcc}
        \toprule
        \multirow{2}{*}{\textbf{Account Level}} & \multicolumn{2}{c}{\textbf{Mods}}                    \\
        \cmidrule{2-3}
        {}                                      & \textbf{Action}                   & \textbf{Account} \\
        \midrule
        \textbf{User}                           & 0                                 & -3               \\
        \textbf{Security}                       & -3                                & -5               \\
        \textbf{Admin}                          & -4                                & -6               \\
        \bottomrule
    \end{tabular}
\end{table}

\subsection{Security Tally}
\index{Security Tally}

\begin{table}[htb]
    \rowcolors{1}{}{lightgray}
    \caption[Security Tally Measures]{Security Tally Measures}
    \label{tab:tally measures}
    \centering
    \begin{tabular}{cl}
        \toprule
        \textbf{Tally} & \textbf{Measure} \\
        \midrule
        \textbf{5}     & Analyze ICE      \\
        \textbf{10}    & Trace ICE        \\
        \textbf{15}    & Silent Alert     \\
        \textbf{20}    & Combat ICE       \\
        \textbf{25}    & Active Alert     \\
        \textbf{50}    & Shutdown         \\
        \bottomrule
    \end{tabular}
\end{table}



\begin{verbatim}
    Tally = abs(d10) + System - Sleaze
\end{verbatim}

\paragraph{Analyze ICE}

Analyze ICE is looking into a deckers activities to find any signs of illegal
actions. If it finds anything it will be added to the deckers security tally.

\paragraph{Trace ICE}

Trace ICE will try to find the deckers location by analyzing its \emph{Stream}.

\paragraph{Passive Alert}
In silent or passive Alert Status a list of predefined personnel is informed
of a possible intrusion. The Node diverts resources to security purposes,
increasing Firewall by 2 and decreasing Processor by 2. Any standard functionality
of the Node could be impaired by this resource transfer (GM discretion).
The information is not broadcasted to Processes in the Node.

\paragraph{Combat ICE} Combat ICE will continuously attack the decker till it
is crashed and restart afterwards to attack again.


\paragraph{Active Alert}

In active Alert Status a list of predefined personnel is informed of an intrusion.
The Node diverts resources to security purposes, increasing Firewall
by 3 and decreasing Processor by 3. Any standard functionality of the Node can
be impaired by this resource transfer (GM discretion).
The information is broadcasted to all Processes in the Node.

\paragraph{Shutdown}

\section{Electronic Warfare}

\paragraph{Find Wireless}
\label{par: find wireless}

\mbox{}\\

\begin{tabular}{rl}
    \textbf{Program}       & Scan                          \\
    \textbf{Prerequisite}  & Target in \emph{Signal} range \\
    \textbf{Test Modifier} & var.                          \\
    \textbf{Duration}      & 10s                           \\
\end{tabular}

\hfill

\paragraph{Jam Wireless}
\label{par: jam wireless}

\mbox{}\\

\begin{tabular}{rl}
    \textbf{Program}       & Scan \\
    \textbf{Prerequisite}  & None \\
    \textbf{Test Modifier} & 0    \\
    \textbf{Duration}      & None \\
\end{tabular}

\hfill
