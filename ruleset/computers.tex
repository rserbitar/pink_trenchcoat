\chapter{Computers}
\label{chap:basics}

This chapter explains both the matrix, including AR and everything computer related
like electronic warfare.


\section{What is the Matrix}

The Matrix is a virtual representation of the cyberspace for human users. It is they way they
perceive interactions between themselves and both other matrix users and \emph{Matrix Entities}.

\subsection{Accessing the Matrix}
\label{subsec:access matrix}

There are various ways to access the matrix.

\paragraph{Physical Access}
This method of matrix access uses outdated methods like keyboard and mouse.
It is generally outdated and very slow. It is only used if people are afraid of any kind of matrix
damage, or are very traditional.

\paragraph{Augmented Reality}
Augmented Reality or AR access is a widely used for of matrix access, especially one the go
or while wanting to do things in parallel. AR users still see the real world, but get additional
information projected on top of it. Thus they can see objects, additional information and also
sound added to the real world that does not exist.

\paragraph{Virtual Reality}
Virtual Reality supersedes the perception of the user. They are not aware of the real world, but
instead see, hear, smell and feal virtual sensory input that is 100\% artificial.

\subparagraph{Tortoise}
Tortoise uses not direct brain interfaces as provided by most data jacks, but uses
outdated technologies like trodes. Due to it not requiring cyberware it is often
used by adepts or magicians.

\subparagraph{Cold Sim}
Cold Sim is the standard way of using the matrix today. The user is experiencing
the matrix by direct stimulation of their sensory cortex so that they see, hear
and feel the matrix. Their thoughts of movements and actions are translated into
commands of their virtual bodies using virtual applications.

\subparagraph{Hot Sim}
Hot Sim is the most dangerous but also the fastest way to access the matrix.
The data is directly fed into the users brain even circumventing their sensory
centers that are stimulated in cold sim. Instead, using knowledge link technology,
the matrix user just instantly knows the information. Also their raw thoughts
are transformed into matrix commands.


\begin{table}[htb]
    \caption[Matrix Access Methods]{Matrix Access Methods}
    \label{tab:matrix access}
    \centering
    \begin{tabular}{cll}
        \toprule
        \textbf{Method}   & \textbf{Input}       & \textbf{Output}         \\
        \midrule
        \textbf{Physical} &
        \makecell[l] {\tabitem Keyboard                                    \\ \tabitem Mouse \\ \tabitem Touchscreen \\ \tabitem Input Trigger} &
        \makecell[l] {\tabitem Screen                                      \\ \tabitem Loudspeaker}        \\
        \midrule
        \textbf{AR}       &
        \makecell[l] {\tabitem Transducer                                  \\ \tabitem Microphone \\ \tabitem AR Gloves \\ \tabitem Holo Scanner} &
        \makecell[l] {\tabitem Lenses                                      \\ \tabitem Vision-Link \\ \tabitem In-Ears \\ \tabitem Sound-Link}        \\
        \midrule
        \textbf{Tortoise} &
        \makecell[l] {\tabitem Trodes                                      \\ \tabitem External \\ Sim Rig} &
        \makecell[l] {\tabitem Trodes                                      \\ \tabitem External \\ Sim Module}        \\
        \midrule
        \textbf{Cold Sim} & \tabitem Sim Rig     & \tabitem Sim Module     \\
        \midrule
        \textbf{Hot Sim}  & \tabitem Transcriber & \tabitem Knowledge Link \\
        \bottomrule
    \end{tabular}
\end{table}

\begin{table}[htb]
    \rowcolors{1}{}{lightgray}
    \caption[Matrix Access Requirements]{Matrix Access Requirements}
    \label{tab:matrix access requirements}
    \centering
    \begin{tabular}{lc}
        \toprule
        \textbf{Method}   & \textbf{Processor}/ \textbf{Uplink} \\
        \midrule
        \textbf{Physical} & 1                                   \\
        \textbf{AR}       & 3                                   \\
        \textbf{Tortoise} & 6                                   \\
        \textbf{Cold Sim} & 6                                   \\
        \textbf{Hot Sim}  & 10                                  \\
        \bottomrule
    \end{tabular}
\end{table}

\begin{table}[htb]
    \rowcolors{1}{}{lightgray}
    \caption[Matrix Access Modifiers]{Matrix Access Modifiers}
    \label{tab:matrix access mods}
    \centering
    \begin{tabular}{lcccc}
        \toprule
        \textbf{Method}   & \textbf{Skill} & \textbf{React} & \textbf{Tick} & \textbf{Damage} \\
        \midrule
        \textbf{Physical} & -3             & -5             & x6            & None            \\
        \textbf{AR}       & -2             & -3             & x3            & Fatigue         \\
        \textbf{Tortoise} & -1             & -2             & x1.5          & Fatigue         \\
        \textbf{Cold Sim} & 0              & 0              & x1            & Stun            \\
        \textbf{Hot Sim}  & +2             & +3             & x0.7          & Physical        \\
        \bottomrule
    \end{tabular}
\end{table}

\section{Matrix building blocks}


\subsection{Matrix Devices}

The Matrix is made up of hardware that is processing and delivering it. Most
notable are are the different pieces of hardware the matrix is running on. In
general four different classes of matrix hardware can be found.

\paragraph{Gadget}
\index{Gadget}
Gadgets are small and cheap pieces of hardware. Some of them are so cheap, they
can be found in throwaway articles like food packaging. Others are powering small
sensors or track positions. They range from pinhead size to coin size.
A typical person is carrying around dozens of them.

\paragraph{Commlink}
\index{Commlink}
Commlinks are not only the most common mans to communicate but also a matrix
hardware class. They are bigger than gadgets, but the smallest of them can fit
into a bigger earring. The standard size is of an average playing card. They
carry enough processing power to allow for at least \emph{Augmented Reality}.

\paragraph{Cyberdeck}
\index{Cyberdeck}
Cyberdecks are a special form factor that only few people need. Much bigger than
a an average commlink, about the size of a shoe-box, they pack much more
processing power. Most cyberdecks are used for illegal purposes and are equipped
with a \emph{Sleaze} module to avoid detection in the matrix.

\paragraph{Mainframe}
\index{Mainframe}
Mainframes are stationary pieces of matrix hardware. They range from shoe-box
size to whole floors of a building. Mainframes are used to service multiple people
or perform high performance computations.


\subsection{Matrix Entities}

Matrix entities are virtual building blocks of the matrix. Although they have a
physical basis, they are purely virtual representations both in virtual- and
augmented reality.

\paragraph{Node}
\index{Node}
\label{par:node}

A Node is a matrix entity with processing power. It has matrix location and can be
\emph{accessed}. A Node can run \emph{Processes}, store \emph{Files} and be the
origin or destination of a \emph{Stream}.

\paragraph{Process}
\index{Process}
\label{par:process}

Processes are matrix entities that actively perform actions. They are running on
their origin \emph{Node}.

\subparagraph{Persona}
\index{Persona}
A Persona is a special kind of \emph{Process} that represents a matrix user and
their actions. \emph{Personae} can \emph{access} \emph{Nodes}. In this case they
are connected to their \emph{origin Node} via a \emph{Stream}.

\subparagraph{Program}
\index{Program}
A program is a piece of software that can be used by a \emph{Persona} or an
\emph{Agent} as a tool to perform various actions. Programs are always attached
to a \emph{Persona} or \emph{Agent}.

\subparagraph{Agent}
\index{Agent}
An agent is a process that can perform autonomous decisions and use \emph{Programs}
to perform actions. \emph{Agents} can \emph{access} \emph{Nodes}. In this case they are
connected to their \emph{origin Node} via a \emph{Stream}.

\subparagraph{ICE}
\index{ICE}
ICE, or Intrusion Countermeasures, are \emph{Agents} with the special purpose to
defend a node from hackers.

\paragraph{Streams}
\index{Stream}
\label{par:stream}

A stream connects two \emph{Nodes}, the origin and the destination, with a data
connection. A stream also connects the \emph{Node} a \emph{Persona} or \emph{Agent}
is running on with the \emph{Node} it is \emph{accessing}.

\paragraph{File}
\index{File}
\label{par:file}

A \emph{File} is a coherent set of any kind of data.

This includes:

\begin{itemize}[parsep=0em]
    \item a text document
    \item a trideo clip
    \item a BTL movie
    \item a voice record
\end{itemize}


\subsection{Access Levels}
\index{Access Level}
\index{Account}

In Pink Trenchcoat a decker that is \emph{accessing} a \emph{Node} is identified
with a given \emph{Access Level}, or \emph{Account}. This \emph{Account} is
specific to the \emph{Node} and linked to the deckers SIN or, in the case of
\emph{Agents}, to their \emph{AID}.


\paragraph{Anonymous}
\index{Anonymous}
\paragraph{User}
\index{User}
\paragraph{Security}
\index{Security}
\paragraph{Admin}
\index{Admin}

\subsection{Matrix Properties}
\paragraph{Access Rights}
\index{Access Rights}

\paragraph{Access ID}
\index{Access ID}
\index{AID}

\paragraph{Subscription List}
\index{Subscription List}

\paragraph{Logs}
\index{Logs}

The \emph{Logs} are a special \emph{File} that contains a history of all
actions in a Node, including all actions of \emph{Personae} and \emph{Agents},
their \emph{AIDs}, the \emph{Files} and \emph{Streams} the created and consumed
and anything else that was done in the \emph{Node}. \emph{Actions} from a
\emph{Process} that has a \emph{Sleaze} rating are only \emph{logged} when
they have been successfully \emph{analyzed} by \emph{Analyze ICE}.

\subsection{Matrix Attributes}

Each \emph{matrix device} has a number of attributes that define its properties in the matrix.

\paragraph{Processor}
\index{Processor}

The \emph{Processor} attribute represents a \emph{Nodes} row computing power. As most
devices are very advanced, a high \emph{Processor} rating is not needed for most every
day tasks. High \emph{Processor} ratings are required for intensive tasks like processing
Sim-Sense signals for example when using \emph{Cold Sim} or the even more complex
\emph{Hot Sim}. The attribute is also useful if a mainframe is supporting a large
user base.

It is also important in matrix combat where combatants try to overwhelm
the opponents \emph{Node}.

\hfill

The \emph{Processor} attribute is mostly related to a \emph{Devices} size. The
bigger a \emph{Device} the higher its rating is on average.

\begin{table}[htb]
    \rowcolors{1}{}{lightgray}
    \caption[Processor Ratings]{Processor Ratings}
    \label{tab:processor ratings}
    \centering
    \begin{tabular}{lc}
        \toprule
        \textbf{Entity}    & \textbf{Processor} \\
        \midrule
        \textbf{Gadget}    & 0-4                \\
        \textbf{Commlink}  & 3-8                \\
        \textbf{Cyberdeck} & 6-13               \\
        \textbf{Mainframe} & 8-21               \\
        \bottomrule
    \end{tabular}
\end{table}


\paragraph{System}
\index{System}

System describes the quality of the operating system and standard software
suite of a \emph{Node}. The higher the ranking the higher the rating of
\emph{Programs} that can be run.

A high Systems rating also helps autonomous software like \emph{ICE} to
perform more efficiently.

\paragraph{Firewall}
\index{Firewall}

Firewall represents the resilience of a \emph{Node} against anything
illegal. This includes any kind of \emph{Exploit} actions leading to
illegal actions not governed by the users level.

Firewall is not determined by a \emph{Nodes} computing power but by
the skill and time invested by the maintainers of the node, and the
number of users and different \emph{Processes} it is supporting.

\hfill

Firewall Ratings are often given by a color coding.

\begin{table}[htb]
    \rowcolors{1}{}{lightgray}
    \caption[Firewall Ratings]{Firewall Ratings}
    \label{tab:firwall ratings}
    \centering
    \begin{tabular}{lc}
        \toprule
        \textbf{Color}        & \textbf{Firewall} \\
        \midrule
        \textbf{Blue}         & 0-4               \\
        \textbf{Green}        & 5-9               \\
        \textbf{Orange}       & 10-14             \\
        \textbf{Red}          & 15-19             \\
        \textbf{Ultra Violet} & 20-21             \\
        \bottomrule
    \end{tabular}
\end{table}

\subparagraph{Blue} Blue \emph{Nodes} represent the lowest level of security.
They are often either very cheap gadgets like Smart Tags or public mainframes
like public libraries.

\subparagraph{Green} Green \emph{Nodes} represent the vast majority of matrix hosts.
They are a good trade-off between expensive security experts and time invest.
\emph{Nodes} with fewer users tend to have higher green ratings.

\subparagraph{Orange} Orange \emph{Nodes} are used when higher security is required,
like in the mainframe of a police station, a law firm, or the \emph{Nodes} of upper
class individuals.

\subparagraph{Red} Red \emph{Nodes} are mostly used by high security facilities like
corporate research sites or  government agencies.

\subparagraph{Ultra Violet} Ultra Violet \emph{Nodes}, if they exist, are only used
for legendary and top-secret institutions.


\paragraph{Uplink}
\index{Uplink}

Uplink describes the quality, speed and volume of data that a \emph{Node}
can access per time. A high throughput is required for \emph{Cold Sim} and even
more for \emph{Hot Sim}. Uplink mostly degrades over distance, although not as
fast as wireless \emph{Signal} does, or if the signal has to go through wireless
channels.

\paragraph{Signal}
\index{Signal}

The Signal rating describes the power and quality of a wireless signal. It is
used to check how far a signal penetrates and also represents the power delivered
in case of \emph{Electronic Warfare}.
Only nodes with wireless capabilities have a Signal rating.

\begin{table}[htb]
    \rowcolors{1}{}{lightgray}
    \caption[Signal Ranges]{Signal Ranges}
    \label{tab:signal ranges}
    \centering
    \begin{tabular}{clcl}
        \toprule
        \textbf{Signal} & \textbf{Range} & \textbf{Signal} & \textbf{Range} \\
        \midrule
        \textbf{0}      & 1 m            & \textbf{11}     & 5 km           \\
        \textbf{1}      & 2 m            & \textbf{12}     & 10 km          \\
        \textbf{2}      & 5 m            & \textbf{13}     & 20 km          \\
        \textbf{3}      & 10 m           & \textbf{14}     & 50 km          \\
        \textbf{4}      & 20 m           & \textbf{15}     & 100 km         \\
        \textbf{5}      & 50 m           & \textbf{16}     & 200 km         \\
        \textbf{6}      & 100 m          & \textbf{17}     & 500 km         \\
        \textbf{7}      & 200 m          & \textbf{18}     & 1,000 km       \\
        \textbf{8}      & 500 m          & \textbf{19}     & 2,000 km       \\
        \textbf{9}      & 1 km           & \textbf{20}     & 5,000 km       \\
        \textbf{10}     & 2 km           & \textbf{21}     & 10,000 km      \\
        \bottomrule
    \end{tabular}
\end{table}

\paragraph{Sleaze}
\index{Sleaze}

Only devices equipped with with an illegal \emph{Sleaze} module have a
\emph{Sleaze} rating. The \emph{Sleaze} rating allows a decker to hide from
security software of a \emph{Node}.
Without it the decker would instantly be recognized after performing any kind of
\emph{Exploit} action.

A \emph{Sleaze} module allows also to broadcast and change (fake) SINs the decker
possesses. The decker can not mimic arbitrary SINs.

\section{Matrix Actions}

\begin{table*}[t]
    %\rowcolors{3}{}{lightgray}
    \caption[Matrix Actions]{Matrix Actions}
    \label{tab:matrix actions}
    \centering
    \begin{tabular}{ccllll}
        \toprule
        \textbf{Account Level} & \textbf{Program} & \textbf{Node}                                                     & \textbf{Process}                                                & \textbf{Stream}                          & \textbf{File}                            \\
        \midrule
        \multirow{5}{*}[-2em]
        {\textbf{Anonymous}}   & None             & \tabitem \hyperref[par:access node]{Anonymous Access}             &                                                                 &                                          &                                          \\
        \cmidrule{3-6}
                               & Analyze          & \tabitem \hyperref[par:analyze]{Analyze}                          & \tabitem \hyperref[par:analyze]{Analyze}                        & \tabitem \hyperref[par:analyze]{Analyze} & \tabitem \hyperref[par:analyze]{Analyze} \\
        \cmidrule{3-6}
                               & Break            &                                                                   &                                                                 & \tabitem \hyperref[par:break]{Break }    & \tabitem \hyperref[par:break]{Break }    \\
        \cmidrule{3-6}
                               & Corrupt          & \makecell[l]{\tabitem \hyperref[par:crash]{Crash}                                                                                                                                                                         \\ \tabitem \hyperref[par:slow node]{Slow}} & \makecell[l]{\tabitem \hyperref[par:crash]{Crash} \\ \tabitem \hyperref[par:slow process]{Slow}} & \tabitem \hyperref[par:corrupt action]{Corrupt} & \tabitem \hyperref[par:corrupt action]{Corrupt} \\
        \cmidrule{3-6}
                               & Find             & \tabitem Find                                                     & \tabitem Find                                                   & \tabitem Find                            & \tabitem Find                            \\
        \cmidrule{2-6}

        \multirow{5}{*}[-2em]
        {\textbf{User}}        & None             & \tabitem \hyperref[par:access node]{User Access}                  & \makecell[l]{ \tabitem \hyperref[par:command (matrix)]{Command}                                                                                       \\                                                                                                         \tabitem \hyperref[par:start process]{Start}                                                                                        \\ \tabitem \hyperref[par:stop process]{Stop}} & \makecell[l]{\tabitem \hyperref[par:decrypt]{Decrypt} \\ \tabitem \hyperref[par:read stream]{Read} \\ \tabitem \hyperref[par:start stream]{Start} \\  \tabitem \hyperref[par:send]{Send} \\ \tabitem \hyperref[par:terminate]{Terminate} } & \makecell[l]{\tabitem \hyperref[par:create file]{Create} \\ \tabitem \hyperref[par:decrypt]{Decrypt} \\ \tabitem \hyperref[par:delete file]{Delete} \\ \tabitem \hyperref[par:read file]{Read} \\ \tabitem \hyperref[par:write file]{Write}} \\
        \cmidrule{3-6}
                               & Control          &                                                                   & \tabitem Control [Thing]                                        &                                          &                                          \\
        \cmidrule{3-6}
                               & Crypt            &                                                                   &                                                                 & \tabitem \hyperref[par:encrypt]{Encrypt} & \tabitem \hyperref[par:encrypt]{Encrypt} \\
        \cmidrule{3-6}
                               & Generate         &                                                                   &                                                                 & \tabitem Generate                        & \tabitem Generate                        \\
        \cmidrule{3-6}
                               & Medic            & \tabitem \hyperref[par:repair]{Repair}                            & \tabitem \hyperref[par:repair]{Repair}                          &                                          &                                          \\
        \cmidrule{2-6}
        \textbf{Security}      & None             & \makecell[l]{\tabitem \hyperref[par:access node]{Security Access}                                                                                                                                                         \\ \tabitem \hyperref[par:view accounts]{View Accounts} \\  \tabitem \hyperref[par:view alert status]{View Alert Status} \\ \tabitem \hyperref[par:view logs]{View Logs} \\ \tabitem \hyperref[par:view subscriptions]{View Subscriptions} } & \makecell[l]{\tabitem Command ICE \\ \tabitem Start ICE \\ \tabitem Stop ICE} & & \\
        \cmidrule{2-6}
        \textbf{Admin}         & None             & \makecell[l]{\tabitem \hyperref[par:access node]{Admin Access}                                                                                                                                                            \\ \tabitem \hyperref[par:change alert status]{Change Alert Status} \\  \tabitem \hyperref[par:edit accounts]{Edit Accounts} \\ \tabitem \hyperref[par:edit logs]{Edit Logs} \\ \tabitem \hyperref[par:edit subscriptions]{Edit Subscriptions} \\ \tabitem Shutdown} &  & &  \\
        \bottomrule
    \end{tabular}
\end{table*}

\subsection{Basic Actions}

\emph{Basic Actions} are very simple and normally do not require a \emph{Test}
or \emph{Program}. If a \emph{Test} is required because the \emph{Character} is
wounded or has an extreme non-technical background use:

\begin{equation*}
    \textit{Computers} +3
\end{equation*}

\myparagraph{Access Node}
\index{Access}
\label{par:access node}

\begin{tabular}{rl}
    \textbf{Prerequisite} & \emph{Node AID} \\
    \textbf{Duration}     & 0.1s            \\
\end{tabular}

\hfill

This action is required to access a \emph{Node} with a known \emph{AID}. After a
successful \emph{Access Action} the decker has \emph{accessed} the \emph{Node}.

Having \emph{accessed} a \emph{Node} is often a prerequisite for lots of
\emph{Matrix Actions} targeting \emph{Files} and \emph{Streams}. It is only of
particular relevance when a decker does not have the relevant \emph{Access Rights}
to \emph{access} the \emph{Node} and needs to \emph{Exploit} their way in.

\myparagraph{Change Alert Status}
\label{par:change alert status}


\begin{tabular}{rl}
    \textbf{Prerequisite} & \emph{Accessed} \emph{Node} \\
    \textbf{Duration}     & 0.5s                        \\
\end{tabular}

\hfill

This action allows the decker to change the \emph{Nodes} \emph{Alert Status}.

\myparagraph{Command}
\index{Command (Matrix)}
\label{par:command (matrix)}

\begin{tabular}{rl}
    \textbf{Prerequisite} & \emph{Process AID}, \emph{Accessed origin Node} \\
    \textbf{Duration}     & 2s                                              \\
\end{tabular}

\hfill

This action allows a decker to give commands to a \emph{Process}. This can either
be an \emph{Agent}, or any other \emph{Program} on a \emph{Node} or \emph{Device}
like a drone or a security camera.

The decker needs the \emph{AID} of the \emph{Process} and needs to \emph{access}
the origin or target \emph{Node} of the \emph{Process}.

\myparagraph{Create File}
\index{Create File}
\label{par:create file}

\begin{tabular}{rl}
    \textbf{Prerequisite} & \emph{Accessed Node} \\
    \textbf{Duration}     & 1s                   \\
\end{tabular}

\hfill

This action creates a \emph{File} in a \emph{Node}. The creator chooses content and
\emph{Access Rights} and gets the \emph{Files} \emph{AID}.

\myparagraph{Decrypt}
\label{par:decrypt}

\begin{tabular}{rl}
    \textbf{Prerequisite} & \emph{Red File}, \emph{CryptKey} \\
    \textbf{Duration}     & 0.1s                             \\
\end{tabular}

\hfill

\emph{Decrypt} and \emph{encrypted} \emph{File} if the decker has the \emph{CryptKey}.

\myparagraph{Delete File}
\label{par:delete file}

\begin{tabular}{rl}
    \textbf{Prerequisite} & \emph{File AID}, \emph{Accessed Node} \\
    \textbf{Duration}     & 0.1s                                  \\
\end{tabular}

\hfill

\emph{Delete} a \emph{File} in a \emph{Node}. After the \emph{File} is
\emph{deleted} it can not be recovered.

\myparagraph{Edit Accounts}
\label{par:edit accounts}


\begin{tabular}{rl}
    \textbf{Prerequisite} & \emph{Accessed} \emph{Node} \\
    \textbf{Duration}     & 0.5s                        \\
\end{tabular}

\hfill

This action allows the decker to edit \emph{Accounts} of a \emph{Nodes}. This
includes removing, adding and changing \emph{Access Levels}. In the case of adding
a new \emph{Accounts} the respective SIN is required.

\myparagraph{Edit Logs}
\label{par:edit logs}


\begin{tabular}{rl}
    \textbf{Prerequisite} & \emph{Accessed} \emph{Node} \\
    \textbf{Duration}     & 0.5s                        \\
\end{tabular}

\hfill

This action allows the decker to edit the \emph{Logs} of a \emph{Node}. This
includes adding and removing entries.

\myparagraph{Edit Subscriptions}
\label{par:edit subscriptions}


\begin{tabular}{rl}
    \textbf{Prerequisite} & \emph{Accessed} \emph{Node}    \\
                          & \emph{Admin Access other Node} \\
    \textbf{Duration}     & 0.5s                           \\
\end{tabular}

\hfill

This action allows the decker to edit the \emph{Subscription List} of a \emph{Nodes}.
This includes removing and adding \emph{Nodes}. In the case of adding
a decker needs Admin Access on the other \emph{Node}.

\myparagraph{Read File}
\label{par:read file}

\begin{tabular}{rl}
    \textbf{Prerequisite} & \emph{File AID}, \emph{Accessed Node} \\
    \textbf{Duration}     & 0.1s                                  \\
\end{tabular}

\hfill

This action allows a decker to read \emph{Files} in a \emph{Node}. \emph{Reading}
a \emph{File} enables a decker to \emph{create} a local \emph{File} copy in the
\emph{Personas} origin \emph{Node}.

\myparagraph{Read Stream}
\label{par:read stream}

\begin{tabular}{rl}
    \textbf{Prerequisite} & \emph{Stream AID}                  \\
                          & \emph{Accessed origin/target Node} \\
    \textbf{Duration}     & 0.1s                               \\
\end{tabular}

\hfill

This action allows a decker to read \emph{Streams} in a \emph{Node}.
\emph{Reading} a \emph{Stream} enables a decker to \emph{create} a local \emph{File}
containing the content of the \emph{Stream} in the \emph{Personas} origin
\emph{Node}.

\myparagraph{Start Process}
\label{par:start process}

\begin{tabular}{rl}
    \textbf{Prerequisite} & \emph{Accessed Node} \\
    \textbf{Duration}     & 1s                   \\
\end{tabular}

\hfill

This action creates a \emph{Process} in a \emph{Node}. The creator chooses its
\emph{Access Rights} and gets the \emph{Process} \emph{AID}.

\myparagraph{Send to Stream}
\label{par:send}

\begin{tabular}{rl}
    \textbf{Prerequisite} & \emph{Stream AID}           \\
                          & \emph{Accessed origin Node} \\
    \textbf{Duration}     & 1s                          \\
\end{tabular}

\hfill

This action creates a \emph{Stream} between two \emph{Nodes}. The creator chooses
content and \emph{Access Rights}.

\myparagraph{Start Stream}
\label{par:start stream}

\begin{tabular}{rl}
    \textbf{Prerequisite} & \emph{Accessed origin Node}      \\
                          & \emph{Accessed destination Node} \\
    \textbf{Duration}     & 1s                               \\
\end{tabular}

\hfill

This action creates a \emph{Stream} between two \emph{Nodes}. The creator chooses
content and \emph{Access Rights} and gets the \emph{Streams} \emph{AID}.

\myparagraph{Stop Process}
\label{par:stop process}

\begin{tabular}{rl}
    \textbf{Prerequisite} & \emph{Process AID}          \\
                          & \emph{Accessed origin Node} \\
    \textbf{Duration}     & 1s                          \\
\end{tabular}

\hfill

This action \emph{stops} a \emph{Process}. A related \emph{Agent} or \emph{Persona}
is instantly shut down.


\myparagraph{Terminate Stream}
\label{par:terminate}


\begin{tabular}{rl}
    \textbf{Prerequisite} & \emph{Stream AID}           \\
                          & \emph{Accessed origin Node} \\
    \textbf{Duration}     & 0.5s                        \\
\end{tabular}

\hfill

This action \emph{terminates} a \emph{Stream}. A related \emph{Process}
is instantly stopped.

\myparagraph{View Accounts}
\label{par:view accounts}


\begin{tabular}{rl}
    \textbf{Prerequisite} & \emph{Accessed} \emph{Node} \\
    \textbf{Duration}     & 0.5s                        \\
\end{tabular}

\hfill

This action allows the decker to view all \emph{User}, \emph{Security} and
\emph{Admin} \emph{Accounts} for the \emph{Node}.

\myparagraph{View Alert Status}
\label{par:view alert status}


\begin{tabular}{rl}
    \textbf{Prerequisite} & \emph{Accessed} \emph{Node} \\
    \textbf{Duration}     & 0.5s                        \\
\end{tabular}

\hfill

This action allows the decker to view the current \emph{Alert Status} of the
\emph{Node}.

\myparagraph{View Logs}
\label{par:view logs}


\begin{tabular}{rl}
    \textbf{Prerequisite} & \emph{Accessed} \emph{Node} \\
    \textbf{Duration}     & 0.5s                        \\
\end{tabular}

\hfill

This action allows the decker to view the current \emph{Logs} of the
\emph{Node}.

\myparagraph{View Subscriptions}
\label{par:view subscriptions}


\begin{tabular}{rl}
    \textbf{Prerequisite} & \emph{Accessed} \emph{Node} \\
    \textbf{Duration}     & 0.5s                        \\
\end{tabular}

\hfill

This action allows the decker to view the \emph{AIDs} of the \emph{Nodes} the are
\emph{subscribed} to the \emph{Node}.

\myparagraph{Write to File}
\label{par:write file}


\begin{tabular}{rl}
    \textbf{Prerequisite} & Found \emph{File} \\
    \textbf{Duration}     & 0.5s              \\
\end{tabular}

\hfill

This action allows a decker to \emph{write} any content to a \emph{File}.

\subsection{Advanced Actions}

\emph{Advanced Actions} require \emph{Tests} to perform and require a \emph{Program}
to carry out. The standard format of a \emph{Test} is given as:

\begin{equation*}
    \textit{Skill}(\textit{Program}) + \textit{Test Modifier}
\end{equation*}

\myparagraph{Analyze [Node, Process, Stream, File]}
\label{par:analyze}


\begin{tabular}{rl}
    \textbf{Program}       & Computer(Analyze)                   \\
    \textbf{Prerequisite}  & Found [Node, Process, Stream, File] \\
    \textbf{Test Modifier} & - Target \emph{Sleaze}              \\
    \textbf{Duration}      & 2s                                  \\
\end{tabular}

\hfill

This action allows for analyzing properties of various matrix entities. To analyze a \emph{Node}
an AID is required. Other entities have to be \emph{found}. \emph{Processes} and \emph{Streams}
can only be analyzed if the decker has \emph{accessed} either the target or the destination
\emph{Node}.

\begin{table}[htb]
    \rowcolors{1}{}{lightgray}
    \caption[Analyze Node Results]{Analyze Node Results}
    \label{tab:analyze results}
    \centering
    \begin{tabular}{cll}
        \toprule
        \textbf{Result} & \textbf{Properties}     & \textbf{Location} \\
        \midrule
        \textbf{0}      & Active Alert Status     &                   \\
        \textbf{2}      & AID                     &                   \\
        \textbf{4}      & Type                    &                   \\
        \textbf{6}      & High/Low Attributes     & Continent         \\
        \textbf{8}      & Functionality           & State             \\
        \textbf{10}     & High/Med/Low Attributes & City              \\
        \textbf{12}     & Active Processes        & Suburb            \\
        \textbf{14}     & Exact Attributes        & Street            \\
        \textbf{16}     &                         & Building          \\
        \textbf{18}     &                         & Room              \\
        \textbf{20}     &                         & Exact             \\
        \bottomrule
    \end{tabular}
\end{table}


\myparagraph{Control}
\label{par:control}

\begin{tabular}{rl}
    \textbf{Program}       & Skill(Control)              \\
    \textbf{Prerequisite}  & \emph{Accessed} \emph{Node} \\
                           & \emph{Process} \emph{AID}   \\
    \textbf{Test Modifier} & var.                        \\
    \textbf{Duration}      & 1s                          \\
\end{tabular}

\hfill

Using the \emph{Control Action} the decker can use any kind of item that can be
\emph{controlled} remotely from a \emph{Process} in a \emph{Node}. The decker has
to use the relevant \emph{Skill} limited by the \emph{Control Program}.

\begin{equation*}
    \textit{Skill(Control)}
\end{equation*}

Examples are using remotely controlled guns using \emph{Gunnery} or driving a
remotely controlled car using \emph{Wheeled}.

\myparagraph{Encrypt}
\label{par:encrypt}

\begin{tabular}{rl}
    \textbf{Program}       & Crypt                                \\
    \textbf{Prerequisite}  & \emph{Accessed} \emph{Node}          \\
                           & \emph{File AID} or \emph{Stream AID} \\
    \textbf{Test Modifier} & 0                                    \\
    \textbf{Duration}      & 1s                                   \\
\end{tabular}

\hfill

The \emph{Encrypt Action} encrypts a \emph{File} or \emph{Stream} so that even
deckers with the required \emph{Access Rights} can not read the content. The
\emph{Action} does not require a \emph{Test} but automatically encrypts the
\emph{File} or \emph{Stream} with the \emph{Crypt Programs} rating.
To read the content one needs either the key or try to \emph{Break} the encryption.


\myparagraph{Find Process}
\label{par:find process}


\begin{tabular}{rl}
    \textbf{Program}       & Computer(Find)                                  \\
    \textbf{Prerequisite}  & \emph{Access} to origin/destination \emph{Node} \\
    \textbf{Test Modifier} & - Target \emph{Sleaze}                          \\
    \textbf{Duration}      & 10s                                             \\
\end{tabular}

\hfill

This action allows a decker to find \emph{Processes} in a \emph{Node}, which must be either its
origin or the destination.

\myparagraph{Find Stream}
\label{par:find stream}


\begin{tabular}{rl}
    \textbf{Program}       & Computer(Find)                                  \\
    \textbf{Prerequisite}  & \emph{Access} to origin/destination \emph{Node} \\
    \textbf{Test Modifier} & - Target \emph{Sleaze}                          \\
    \textbf{Duration}      & 10s                                             \\
\end{tabular}

\hfill

This action allows a decker to find \emph{Streams} in a \emph{Node}, which must be either its
origin or the destination.

\myparagraph{Find File}
\label{par:find file}

\begin{tabular}{rl}
    \textbf{Program}       & Computer(Find)                                  \\
    \textbf{Prerequisite}  & \emph{Access} to origin/destination \emph{Node} \\
    \textbf{Test Modifier} & - Target \emph{Sleaze}                          \\
    \textbf{Duration}      & 10s                                             \\
\end{tabular}

\hfill

This action allows a decker to find \emph{Files} in a \emph{Node}.

\subsection{Matrix Combat}
\label{subsec:matrix combat}

\emph{Actions} in this section directly harm \emph{Matrix Entities} either by
damaging or slowing them. Most of the time, \emph{Processor} is an important
\emph{Attribute} in \emph{Matrix Combat}.

\myparagraph{Corrupt}
\label{par:corrupt action}

\begin{tabular}{rl}
    \textbf{Program}       & Cyber Combat(Corrupt)                \\
    \textbf{Prerequisite}  & \emph{Accessed} \emph{Node}          \\
                           & \emph{File AID} or \emph{Stream AID} \\
                           & or Found [\emph{File}/\emph{Stream}] \\
    \textbf{Test Modifier} & - Originating \emph{Node} System     \\
    \textbf{Duration}      & 1s                                   \\
\end{tabular}

\hfill

If a decker does not have the \emph{Access Right} to \emph{delete} a \emph{File}
or \emph{Terminate} a \emph{Stream} the decker can \emph{corrupt} it so it becomes
unusable.

For each point of \emph{Test Quality} deal \emph{Processor} \emph{Matrix Damage}
to the \emph{File} or \emph{Stream}.

In addition each \emph{Corrupt Test} that can increase the deckers
\emph{Security Tally}.
The increase is given by:

\begin{equation*}
    \textit{Tally} = max(0, \textit{5fN} + \textit{System} - \textit{Sleaze})
\end{equation*}

\myparagraph{Crash}
\label{par:crash}

\begin{tabular}{rl}
    \textbf{Program}       & Cyber Combat(Crash)                \\
    \textbf{Prerequisite}  & Found [\emph{Node}/\emph{Process}] \\
    \textbf{Test Modifier} & - System                           \\
    \textbf{Duration}      & 1s                                 \\
\end{tabular}

\hfill

For each point of \emph{Test Quality} deal \emph{Processor} \emph{Matrix Damage}
to the \emph{Node} or \emph{Process}.

In addition each \emph{Crash Test} that is targeting a \emph{Node}
can increase the deckers \emph{Security Tally}.
The increase is given by:

\begin{equation*}
    \textit{Tally} = max(0, 2 * \textit{5fN} + \textit{System} - \textit{Sleaze})
\end{equation*}

\myparagraph{Repair}
\label{par:repair}


\begin{tabular}{rl}
    \textbf{Program}       & Computer(Medic) \\
    \textbf{Prerequisite}  & AID             \\
    \textbf{Test Modifier} & 0               \\
    \textbf{Duration}      & 10s             \\
\end{tabular}

\hfill

The \emph{Repair Action} allows a decker to \emph{repair} \emph{Matrix Damage}
on \emph{Nodes}, \emph{Processes}, \emph{Files} and \emph{Streams}.
For each point of \emph{Test Quality} repairs one point of \emph{Matrix Damage}
to the target.


\myparagraph{Slow Node}
\label{par:slow node}


\begin{tabular}{rl}
    \textbf{Program}       & Cyber Combat(Slow)           \\
    \textbf{Prerequisite}  & \emph{Access} to \emph{Node} \\
    \textbf{Test Modifier} & - System                     \\
    \textbf{Duration}      & 1s                           \\
\end{tabular}

\hfill

The \emph{Slow Action} allows a decker to reduce the \emph{Processor} a target
\emph{Node}. Reduce the \emph{Processor} by the \emph{Test Quality}. This effect
lasts for 10s, or till the decker takes another \emph{Action}, whichever happens
later.

\myparagraph{Slow Process}
\label{par:slow process}


\begin{tabular}{rl}
    \textbf{Program}       & Cyber Combat(Slow)          \\
    \textbf{Prerequisite}  & Found \emph{Process} or AID \\
    \textbf{Test Modifier} & - System                    \\
    \textbf{Duration}      & 1s                          \\
\end{tabular}

\hfill

The \emph{Slow Action} allows a decker to increase the \emph{Tick Cost} of
\emph{Actions} a target \emph{Process}, including \emph{Personae} and \emph{Agents}.
Increase Free, Simple, Half, Full Actions by 1,2,3,4 times the \emph{Test Quality}.
Longer \emph{Action} get increased by 50\% for each point of \emph{TQ}.
This effect lasts till the targets next \emph{Action}, or till the decker takes
another \emph{Action}, whichever happens later.

\subsection{Hacking}

\myparagraph{Break}
\label{par:break}

\begin{tabular}{rl}
    \textbf{Program}       & Hacking(Break)                               \\
    \textbf{Prerequisite}  & \emph{Read File}                             \\
    \textbf{Test Modifier} & - Crypt Rating -3                            \\
    \textbf{Duration}      & $10 * \textit{max}(1, \textit{5fN})$ seconds \\
\end{tabular}

\hfill

A successful \emph{Break} \emph{Test} removes the \emph{Encryption} from a
\emph{File}. The \emph{Duration} is determined by the \emph{5fN} roll of the
\emph{Game Master} to a minimum of 10 seconds.

\paragraph{Exploit}
\index{Exploit}

Every time a decker wants to perform an \emph{Action} where their \emph{Access Level}
is not high enough, like \emph{Viewing} the security \emph{Log} without being
at least \emph{Security} level, an \emph{Exploit} \emph{Test} is required.
If the \emph{Action} in question requires a \emph{Test} itself,
when for example \emph{Writing to a Stream}, the \emph{Exploit Test} does not
replace the actual \emph{Test} but is an additional requirement.

An Exploit test is an opposed test between the deckers \emph{Hacking(Exploit)} and
the \emph{Nodes} \emph{Firewall}.

\begin{equation*}
    \textit{Test Quality} = \textit{10fR} + \textit{Hacking(Exploit)} - \textit{Firewall}
\end{equation*}

In addition each \emph{Exploit Test} can increase the deckers \emph{Security Tally}.
The increase is given by:

\begin{equation*}
    \textit{Tally} = max(0, \textit{5fN} + \textit{System} - \textit{Sleaze})
\end{equation*}

\begin{table}[htb]
    \rowcolors{3}{}{lightgray}
    \caption[Exploit Modifiers]{Exploit Modifiers}
    \label{tab:exploit mods}
    \centering
    \begin{tabular}{lcc}
        \toprule
        \multirow{2}{*}{\textbf{Account Level}} & \multicolumn{2}{c}{\textbf{Mods}}                    \\
        \cmidrule{2-3}
        {}                                      & \textbf{Action}                   & \textbf{Account} \\
        \midrule
        \textbf{User}                           & 0                                 & -3               \\
        \textbf{Security}                       & -3                                & -5               \\
        \textbf{Admin}                          & -4                                & -6               \\
        \bottomrule
    \end{tabular}
\end{table}

\subsection{Related Actions}

\myparagraph{Physical Reboot}
\index{Physical Reboot}

A \emph{Physical Reboot} can only be done while having physical access to the
\emph{Node}. It does not require a test and takes time depending on the
\emph{Processor} of the \emph{Node}:

\begin{equation*}
    \textit{Reboot Time} = \textit{Processor}^2 + \textit{seconds}
\end{equation*}

During a \emph{Physical Reboot} the Admin Account can be changed to whatever
\emph{SIN(s)} or \emph{AID(s)} the person that does the \emph{Reboot} desires.

While the \emph{Node} is rebooting all \emph{Processes} are \emph{Stopped}, all
\emph{Streams} terminated, and all \emph{Subscriptions} are inactive. The
\emph{Node} can not be accessed and all \emph{Processes} \emph{Accessing} the
\emph{Node} are disconnected, maybe resulting in \emph{Dump Shock}.


After the \emph{Reboot} all \emph{Matrix Damage} to \emph{Processes}, \emph{Files}
and \emph{Streams} is removed.

\myparagraph{Jack Out}
\index{Jack Out}

\myparagraph{Data Search}
\index{Data Search}

\subsection{Security Tally}
\index{Security Tally}

\begin{table}[htb]
    \rowcolors{1}{}{lightgray}
    \caption[Security Tally Measures]{Security Tally Measures}
    \label{tab:tally measures}
    \centering
    \begin{tabular}{cl}
        \toprule
        \textbf{Tally} & \textbf{Measure}   \\
        \midrule
        \textbf{5}     & Analyze ICE        \\
        \textbf{10}    & Trace ICE          \\
        \textbf{15}    & Silent Alert       \\
        \textbf{20}    & Combat ICE         \\
        \textbf{25}    & Active Alert       \\
        \textbf{50}    & Emergency Shutdown \\
        \bottomrule
    \end{tabular}
\end{table}


\paragraph{Analyze ICE}

Analyze ICE is looking into a deckers activities to find any signs of illegal
actions. If it finds anything it will be added to the deckers security tally.

\paragraph{Trace ICE}

Trace ICE will try to find the deckers location by analyzing its \emph{Stream}.

\paragraph{Passive Alert}
In silent or passive Alert Status a list of predefined personnel is informed
of a possible intrusion. The Node diverts resources to security purposes,
increasing Firewall by 2 and decreasing Processor by 2. Any standard functionality
of the Node could be impaired by this resource transfer (GM discretion).
The information is not broadcasted to Processes in the Node.

\paragraph{Combat ICE} Combat ICE will continuously attack the decker till it
is crashed and restart afterwards to attack again.


\paragraph{Active Alert}

In active Alert Status a list of predefined personnel is informed of an intrusion.
The Node diverts resources to security purposes, increasing Firewall
by 3 and decreasing Processor by 3. Any standard functionality of the Node can
be impaired by this resource transfer (GM discretion).
The information is broadcasted to all Processes in the Node.

\paragraph{Shutdown}

\section{Electronic Warfare}

\myparagraph{Find Wireless}
\label{par:find wireless}


\begin{tabular}{rl}
    \textbf{Program}       & Scan                          \\
    \textbf{Prerequisite}  & Target in \emph{Signal} range \\
    \textbf{Test Modifier} & var.                          \\
    \textbf{Duration}      & 10s                           \\
\end{tabular}

\hfill

\myparagraph{Jam Wireless}
\label{par:jam wireless}


\begin{tabular}{rl}
    \textbf{Program}       & Scan \\
    \textbf{Prerequisite}  & None \\
    \textbf{Test Modifier} & 0    \\
    \textbf{Duration}      & None \\
\end{tabular}

\hfill
